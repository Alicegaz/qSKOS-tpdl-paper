%!TEX root = main.tex

\section{Introduction}\label{sec:introduction}

% BACKGROUND

% What is SKOS?
The Simple Knowledge Organization System (SKOS) \cite{Miles2005} is a standard model for sharing and linking controlled vocabularies on the Web. Many organizations, including the European Union\footnote{EuroVoc, \url{http://eurovoc.europa.eu/}}, the United Nations\footnote{AGROVOC, \url{http://aims.fao.org/website/AGROVOC-Thesaurus/sub}}, or the UK government\footnote{Integrated Public Sector Vocabulary (IPSV), \url{http://doc.esd.org.uk/IPSV}} have created SKOS representations of their controlled vocabularies and published them on the Web so that they can easily be accessed and by humans and machines.

% What does "quality" mean in this context?
Each vocabulary reflects domain- and application-specific needs and vocabulary designers usually apply their own guidelines and rules (cf.,~\cite{Coronado2009}\todo{CM}{is this correct?}) to meet certain quality requirements. However, these requirements can differ across domains and lead to quality problems when a vocabulary is used in other environments. As a consequence, currently available vocabularies are, even though they are expressed in SKOS, often heterogeneous in terms of quality. The following examples illustrate that this can affect their applicability for tasks like query expansion, faceted browsing, clustering, or auto-completion: 

\begin{itemize}

	\item The SKOS representation of Agrovoc defines concepts in 25 different languages. However, while most concepts have an English description, only 38\% of all concepts have German labels attached. This can be a problem for applications that use Agrovoc and rely on German translations.

	\item The previous version of the STW thesaurus, contained 5 pairs of concepts with identical labels. As a result, the auto-complete function of the online search interface suggested identical entries, without giving disambiguation information.

% Note: examples related to "defense / army & co" are always a bit tricky; maybe we find a better example. 

	\item The vocabulary developed by the Austrian Armed Forces (LVAk thesaurus\footnote{Non-public thesaurus developed by the Austrian Armed Forces}) contains 11 disconnected concept clusters. When we confronted the thesaurus maintainers with these structures, they recognized them as ``forgotten'' test data that has no practical significance.
	
% Note: I am still having problems to imagine why and how one would implement query expansion based on the DBPedia categories...as long as this is not clear, better skip the example.
% mein fantasie-szenario ist folgendes: jmd hat einen korpus an dokumenten und will diesen nach kategorien indizieren. also sucht man ein existierendes vocab das man verwenden könnte (irgendwelche subject headings oder eben die dbpedia kategorien weil die eben schon in SKOS verfügbar sind und dbpedia sowieso das zentrum der ld-welt darstellt :). Also verschlagwortet man den dokumentkorpus mit dbpedia-categories und kommt dann, aufgrund der loose concepts drauf, dass dokumente die in ähnliche kategorien (broader/narrower) fallen, nicht gefunden werden können, also den recall verschlechtern. ist das unrealistisch?

    % \item In the DBpedia category\footnote{\url{http://downloads.dbpedia.org/3.7/en/skos_categories_en.nt.bz2}} hierarchy, more than 10 percent of the categories not linked to any other category. When other vocabularies link to the DBPedia categories, e.g., to enable query expansion, it can negatively affect recall and reduce navigability.

\end{itemize}

% PROBLEM / RESEARCH QUESTION

The SKOS specification doesn't make any explicit statements about vocabulary quality. It defines a set of \textbf{integrity conditions} that state ``whether or not given data are consistent with respect to the SKOS data model'' [\todo{SKOS reference}]. Existing quality checking tools, such as the PoolParty Thesaurus Consistency Checker\footnote{\url{http://demo.semantic-web.at:8080/SkosServices/check}} or SKOSify\footnote{\url{http://code.google.com/p/skosify/}} can be used to verify that a given vocabulary fulfills these conditions. However, these conditions fail to capture \textbf{quality criteria} such as those mentioned in the examples before; criteria that go beyond the SKOS data model. Such criteria might be part of design rules and guidelines, but rely on \textbf{human judgement}, which is subjective and does not scale for larger, possibly linked Web vocabularies.

% CONTRIBUTION

The goal of our work is to provide a set of quantitative quality indicators for SKOS vocabularies. We believe that this can help domain experts in judging the quality of existing or newly created vocabularies and provide important feedback to the overall vocabulary design process. The main contributions of this work can be summarized as follows:

\begin{itemize}

	\item We identified \textbf{quality criteria} for controlled vocabularies by examining existing literature, tools, and public mailing list discussions. Then we \textbf{formalized} them into computable \textbf{quality metrics} so that they can be implemented consistently by different tools.
	
	\item We implemented these metrics in the \textbf{qSKOS quality assessment tool}.

	\item We \textbf{analyzed} a representative \textbf{set of existing SKOS vocabularies} to learn about possible quality issues in existing Web vocabularies.

\end{itemize}

The results from our studies show that... \todo{BH,CM}{Lorem ipsum dolor sit amet, consectetur adipisicing elit, sed do eiusmod tempor incididunt ut labore et dolore magna aliqua. Ut enim ad minim veniam, quis nostrud exercitation ullamco laboris nisi ut aliquip ex ea commodo consequat. Duis aute irure dolor in reprehenderit in voluptate velit esse cillum dolore eu fugiat nulla pariatur. Excepteur sint occaecat cupidatat non proident, sunt in culpa qui officia deserunt mollit anim id est laborum.}

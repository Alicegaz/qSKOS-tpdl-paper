%!TEX root = main.tex
\section{Conclusions and Future Work}\label{sec:conclusions}

In this paper we identified and defined quality issues of SKOS vocabularies and implemented them in the qSKOS quality assessment tool, which can find possible affected resources in a given SKOS vocabulary. We also analyzed a representative set of existing SKOS vocabularies and found real-world occurrences for all these issues. 

Among the most omnipresent issues we found during analyzing the chosen set of vocabularies were missing or incomplete language information as well as orphan concepts. It also turned out that there is room for a more intense interlinking of SKOS vocabularies.

During our work on qSKOS and while writing this paper we were in contact with some of the creators of the vocabularies we analyzed and also reported back initial results from our analysis. At the time of this writing, we know that our findings lead to improvements in at least two SKOS vocabularies.

We are aware that the quality issues we described in this work are purely quantitative indicators that, on their own, cannot be generalized into statements about the quality of a vocabulary. To learn more about the real-world impact of our work, we would like to conduct a qualitative study in which we confront taxonomists with our results. We will also further collect community feedback, enhance the issues list, and set up a Web-based SKOS quality checking service.


% \section{Acknowledgements}

% Tom Dent (esd support team) for providing a dump for the IPSV vocabulary
% Andrew Gibson for providing a dump for the Peroxisome knowledge Base and giving valuable feedback on their vocabulary structure.

%!TEX root = main.tex

\section{Analysis of Existing SKOS Vocabularies}\label{sec:analysis}

We used the qSKOS quality assessment tool to find possible quality issues in existing SKOS vocabularies. From each quality checking function we obtained detailed reports listing possibly affected resources.

\subsection{Vocabulary Data Set}

Table~\ref{tab:vocabs} summarizes some basic statistical properties of our vocabulary selection: the number of concepts and authoritative concepts, all \texttt{skos:prefLabel}, \texttt{skos:altLabel}, and \texttt{skos:hiddenLabel} relations involving concepts (\texttt{Concept Labels}), all asserted semantic relations, as well as the number of concept schemes. From these properties we can, for instance, already see that approximately 3,000 \texttt{DBpedia Categories} concepts do not have labels (e.g., \url{Category:South_Korean_social_scientists}), which is an indicator for missing natural language descriptions in some Wikipedia categories. 

\begin{table}
\caption{Analyzed SKOS vocabularies}
\label{tab:vocabs}
\begin{center}
\resizebox{\textwidth}{!} {
\setlength{\extrarowheight}{5pt}

\begin{tabular}{p{6cm}ccccccccc}

\textbf{Vocabulary} & \rotatebox{90}{\textbf{Abbreviation}} & \rotatebox{90}{\textbf{\parbox{2.5cm}{Version/\\Last Modified}}} & \rotatebox{90}{\textbf{Concepts}} & \rotatebox{90}{\textbf{\parbox{2.5cm}{Authoritative\\Concepts}}} & \rotatebox{90}{\textbf{\parbox{2.5cm}{Concept\\Labels}}} & \rotatebox{90}{\textbf{\parbox{2.5cm}{Semantic\\Relations}}} & \rotatebox{90}{\textbf{\parbox{2.53cm}{Concept\\Schemes}}} \\
\toprule
United Nations Agricultural Thesaurus & \textbf{AGROVOC} & 1.3 & 32,035 & 32,035 & 620,629 & 65,934 & 1 \\
\hline
DBpedia Categories & \textbf{DBpedia} & 3.7 & 743,410 & 743,410 & 740,352 & 1,490,316 & 0 \\
\hline
The EU's Multilingual Thesaurus & \textbf{Eurovoc} & 5.0 & 6,797 & 6,797 & 457,788 & 18,491 & 128 \\
\hline
Geonames Ontology & \textbf{Geonames} & 2.2.1 & 671 & 671 & 671 & 0 & 9 \\
\hline
Gemeenschappelijke Thesaurus Audiovisuele Archieven & \textbf{GTAA} & 2010/08/25 & 171,991 & 171,991 & 178,776 & 50,892 & 9 \\
\hline
Integrated Public Sector Vocabulary & \textbf{IPSV} & 2.00 & 4,732 & 3,080 & 7,945 & 13,843 & 3 \\
\hline
Library of Congress Subject Headings & \textbf{LCSH} & 2012/03/29 & 443,164 & 408,009 & 750,219 & 598,134 & 1 \\
\hline
Austrian Armed Forces Thesaurus & \textbf{LVAk} & 0.9 & 13,411 & 13,411 & 17,250 & 16,346 & 0 \\
\hline
Middle Kingdom Tombs of Ancient Egypt Thesaurus & \textbf{Meketre} & 2011/07/07 & 422 & 422 & 569 & 1,698 & 2 \\
\hline
Medical Subject Headings & \textbf{MeSH} & \cite{van2006method} & 24,626 & 24,626 & 150,617 & 38,858 & 0 \\
\hline
North American Industry Classification System & \textbf{NAICS} & 2012 & 4,175 & 2,213 & 0 & 8,684 & 1 \\
\hline
New York Times People & \textbf{NYTP} & 2010/06/22 & 4,979 & 4,979 & 4,979 & 0 & 1 \\
\hline
University of Southampton Pressinfo & \textbf{Pressinfo} & 2011/02/24 & 1,125 & 1,125 & 0 & 0 & 0 \\
\hline
Peroxisome Knowledge Base & \textbf{PXV} & 1.6 & 2,112 & 1,686 & 3,628 & 2,695 & 1 \\
\hline
Thesaurus for Economics & \textbf{STW} & 8.10 & 25,107 & 6,789 & 58,441 & 91,816 & 3 \\
\bottomrule
\end{tabular}
}
\end{center}
\end{table}

\subsection{Results and Discussion}

The results of this analysis are summarized in Table~\ref{tab:results}, which shows the absolute number of possibly affected resources for each quality checking function and vocabulary. Numbers marked with an asterisk (*) were obtained by extrapolating from subsets containing 5\% of the respective vocabulary resources.

\begin{table}[h]
\caption{Results of the quality checking functions}
\label{tab:results}

\begin{center}
\resizebox{\textwidth}{!} {
\setlength{\extrarowheight}{5pt}

\begin{tabular}{lccccccccccccccc}
\textbf{Issue} & \rotatebox{90}{\textbf{AGROVOC}} & \rotatebox{90}{\textbf{DBpedia}} & \rotatebox{90}{\textbf{Eurovoc}} & \rotatebox{90}{\textbf{Geonames}} & \rotatebox{90}{\textbf{GTAA}} & \rotatebox{90}{\textbf{IPSV}} & \rotatebox{90}{\textbf{LCSH}} & \rotatebox{90}{\textbf{LVAk}} & \rotatebox{90}{\textbf{Meketre}} & \rotatebox{90}{\textbf{MeSH}} & \rotatebox{90}{\textbf{NAICS}} & \rotatebox{90}{\textbf{NYTP}} & \rotatebox{90}{\textbf{Pressinfo}} & \rotatebox{90}{\textbf{PXV}} & \rotatebox{90}{\textbf{STW}} \\

\toprule
Omitted or Invalid Language Tags & 0 & 0 & 219 & 0 & 0 & 0 & 100,316 & 13,411 & 0 & 23,950 & 0 & 0 & 1,224 & 1,578 & 2 \\

Incomplete Language Coverage & 32,035 & 0 & 6370 & 0 & 0 & 0 & 0 & 0 & 420 & 0 & 0 & 0 & 0 & 0 & 25,050 \\

Undocumented Concepts & 32,035 & 743,410 & 5,341 & 0 & 96,850 & 4,551 & 342,848 & 13,411 & 422 & 1,807 & 3,259 & 4,094 & 1,125 & 1,918 & 23,752 \\

Label Conflicts & 2,949 & 0 & 48 & 18 & 12,404 & 0 & 10,862 & 13 & 4 & 0 & 0 & 0 & 0 & 7 & 0 \\

\midrule

Orphan Concepts & 0 & 77,062 & 7 & 671 & 162,000 & 0 & 173,149 & 21 & 0 & 0 & 0 & 4,979 & 1,125 & 2 & 70 \\

Weakly Connected Components & 4 & 1,506 & 4 & 0 & 621 & 1 & 22,343 & 11 & 5 & 4 & 1 & 0 & 0 & 10 & 141 \\

Cyclic Hierarchical Relations & 0 & 1,132 & 0 & 0 & 0 & 0 & 0 & 5 & 0 & 4 & 0 & 0 & 0 & 0 & 0 \\

Valueless Associative Relations & 282 & 8,839 & 6 & 0 & 9,448 & 253 & 0 & 5 & 0 & 550 & 0 & 0 & 0 & 0 & 5,004 \\

Solely Transitively Related Concepts & 0 & 0 & 2,652 & 0 & 0 & 0 & 0 & 0 & 36 & 0 & 2,189 & 0 & 0 & 0 & 0 \\

Omitted Top Concepts & 0 & 0 & 1 & 9 & 9 & 0 & 1 & 0 & 0 & 0 & 0 & 1 & 0 & 0 & 2 \\

Top Concept Having Broader Concepts & 0 & 0 & 0 & 0 & 0 & 0 & 0 & 0 & 0 & 0 & 0 & 0 & 0 & 1 & 0 \\

\midrule

Missing In-Links & 32,035 & 733,800 & 6,796 & 19 & 171,980* & 3,080 & 408,000* & 13,411 & 422 & 24,625 & 2,213 & 20 & 1,125 & 1,686 & 6,781 \\

Missing Out-Links & 32,035 & 743,410 & 6,797 & 671 & 171,991 & 0 & 344,054 & 13,411 & 273 & 24,626 & 1 & 0 & 1,116 & 1,046 & 0 \\

Broken Links & 238 & 0* & 0* & 0 & 0 & 1 & 780 & 0 & 425 & 1 & 3,169 & 7 & 11 & 163 & 575 \\

Undefined SKOS Resources & 0 & 0 & 0 & 0 & 0 & 1 & 0 & 0 & 0 & 1 & 0 & 0 & 0 & 0 & 0 \\

\bottomrule
\end{tabular}
}
\end{center}
\end{table}

% LABELING AND DOCUMENTATION ISSUES
We found labeling and documentation issues in all vocabularies.
% Omitted or Invalid Language Tags
\texttt{MeSH}, \texttt{PXV}, \texttt{Pressinfo}, and \texttt{LVAk} omit language tags with their labeling properties, \texttt{LCSH} with the \texttt{skos:note} property. \texttt{STW} does not use language tags with 2 instances of \texttt{skos:definition}.
% Incomplete Language Coverage
\texttt{AGROVOC} covers 25 languages but no single concept is labeled in all languages, in \texttt{Meketre} all concepts have English but only some of them French labels assigned. \texttt{STW}, which is expressed mainly in English and German, has many concepts with incomplete language coverage because it (i) links to non-authoritative concepts that are only labeled in German and (ii) uses the private, but valid language tag \texttt{x-other} with some of its concept labels.
%The reasons for the high number of concepts with incomplete language coverage in \texttt{STW} are that many concepts (i) link to non-authoritative concepts that are only labeld in German and (ii) lack labels tagged with the private, but valid language tag \texttt{x-other} that few concepts pull in. 
% Undocumented Concepts
\texttt{Geonames}, which defines a concept scheme of ``feature codes'', is the only vocabulary in our dataset, which has at least one documentation property assigned to all of its concepts. All other vocabularies have a significant number of undocumented concepts. {\textcolor{red}{A finer-grained evaluation on documentation quality indicators like, e.g., average number or length of documentation statements per concept will be subject of future work.}
% Label Conflicts
We also detected possible label conflicts in half of the vocabularies. \texttt{PXV}, for instance, uses the string ``primary peroxisomal enzyme deficiency'' with two concepts in the same concept scheme, but once with a \texttt{skos:prefLabel} and another time with a \texttt{skos:altLabel} property. In \texttt{NAICS} we could not detect any labeling issues but found that it expresses statements with \texttt{skosxl:prefLabel} as predicate and plain literals as object, which contradicts the SKOS-XL specification. 

%\textcolor{red}{For every concept in \texttt{NAICS}, a \texttt{rdfs:label} as well as a \texttt{skosxl:prefLabel} is defined. However, objects of the \texttt{prefLabel} triples are plain literals which seems to be a misunderstanding of SKOS-XL by the vocabulary creators. Thus, this information is not processed and no conflicts are found.}

% STRUCTURAL ISSUES
When analyzing the vocabularies for structural issues, we found that certain results can be seen as indicators for the types of vocabularies.
% Orphan Concepts
In the \texttt{Pressinfo}, \texttt{Geonames}, and \texttt{NYTP} vocabulary, all concepts are orphan concepts, which means that these vocabularies are authority files rather than thesauri or taxonomies. This also implies that these vocabularies have no weakly connected components. \texttt{GTAA} is a mixture of name authority file (approx.~162K concepts) and thesaurus (approx.~10K concepts). The 70 orphan concepts in \texttt{STW} are deprecated concepts.

%the 21 \texttt{LVAk} concepts are obviously test data. <- CM: we should skip this, the preflabels of the concepts seem OK, I have no explanations why they are there

% Weakly Connected Components
Two vocabularies (\texttt{IPSV}, \texttt{NAICS}) consist of only one ``giant component'', which is often considered as being the ideal vocabulary structure. \texttt{STW} forms one giant component (containing 24,572 concepts), but has also 140 additional weakly connected components, which all contain linked authoritative and non-authoritative concepts. All other vocabularies split into several clusters of semantically related concepts, each of which represents a certain subtopic. \texttt{Eurovoc}, for instance, contains 4 weakly connected components, containing 4, 5, 6 and 6775 concepts. The three smallest components contain concepts thematically related to ``Africa'', ``educational institutions'' and ``forecasts''. Weakly connected components divide the \texttt{Meketre} vocabulary into different topics, e.g., museums or concepts reserved for internal use. \texttt{GTAA} consists of 621 weakly connected components with the largest one being a subject thesaurus containing 8413 concepts. Most of the other components contain less than 10 concepts, representing, e.g., locations, person names, and genres. \texttt{PXV} splits into 10 topic-related weakly connected components, such as ``deficiencies'', ``defects'' or ``signals''. The \texttt{LVAk} thesaurus contains 11 concept clusters, which are obviously forgotten test data. 

% Cyclic Hierarchical Relations
Hierarchical cycles are not a common issue except in the collaboratively created \texttt{DBpedia} vocabulary, where many concepts have reflexive \texttt{skos:broader} relations.
% Valuless Associative Relations
Valueless associative relations occur in 8 vocabularies, with their total number being relatively low compared to the total number of all semantic relations in the respective vocabularies.
% Solely Transitive Related Concepts
Solely transitive related concepts occur in 3 vocabularies, establishing relations using properties that, according to the SKOS reference documentation, should not be asserted directly. This indicates a possible misinterpretation of the SKOS specification and could result in a loss in recall on hierarchical queries when the vocabulary is queried without RDFS inferencing.
% Omitted Top Concepts
\texttt{GTAA} and \texttt{Geonames} omit top concepts in all concept schemes they define. \texttt{Eurovoc} uses 128 concept schemes but has one without top concept, which simply contains all concepts defined in the vocabulary. Such an ``umbrella concept scheme'' without top concept is also present in \texttt{LCSH} and \texttt{NYTP}.
% Top Concepts Having Broader Concepts
Only the \texttt{PXV} vocabulary is affected by top concepts having broader concepts in its current version. In earlier versions more of them could be found which were, according to the vocabulary creator, abandoned but still available in the triple store, probably caused by some bug in the vocabulary management software.

% LINKED DATA SPECIFIC ISSUE

The difference between the number of concepts and the number of authoritative concepts in Table~\ref{tab:vocabs} already indicates which vocabularies are linked with other SKOS vocabularies.
% Missing In-Links
However, except \texttt{NYTP} and \texttt{Geonames}, no vocabulary has a high number of estimated in-links from other web resources.
%Estimation of these numbers is rather inaccurate since currently we only rely on the Sindice index. 
% Missing Out-Links
Also the number of out-links is rather low: \texttt{NYTP}, \texttt{IPSV}, and \texttt{STW} are the three exceptions, which are fully linked to other Web resources. The one concept with missing out-links in \texttt{NAICS} is the object of a \texttt{skos:broaderTransitive} relation. One reason for a high number of missing out-links is that the links were not available in the main thesaurus file, which is at least the case for \texttt{AGROVOC}.
% Broken Links
Even though we could not determine the exact number of broken links because of the large number of links to resolve  (over 400K in \texttt{Eurovoc}, over 500K in \texttt{LCSH}), we found that broken links are a common issue in most vocabularies.
% Undefined SKOS Resources
Undefined SKOS resources seem to be a minor issue, because we could only find two of them in all vocabularies: \texttt{MeSH} introduces \texttt{skos:annotation} and \texttt{IPSV} uses the deprecated \texttt{skos:prefSymbol} property.

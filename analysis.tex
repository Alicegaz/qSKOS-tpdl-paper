%!TEX root = main.tex

\section{Analysis of Existing SKOS Vocabularies}\label{sec:analysis}

We used the qSKOS quality assessment tool to find possible quality issues in existing SKOS vocabularies. From each quality checking function we obtained detailed reports listing possibly affected resources.

\subsection{Vocabulary Data Set}

The basic statistical properties of our vocabulary selection are summarized in Table~\ref{tab:vocabs}: the number of concepts, ``authoritative concepts'', as explained in Section~\ref{subsec:ld_issues}, all \texttt{skos:prefLabel}, \texttt{skos:altLabel}, and \texttt{skos:hiddenLabel} relations (\texttt{Labels}), all asserted semantic relations, as well as the number of concept schemes. From these properties we can, for instance, already see that approximately 3,000 \texttt{DBPedia Categories} concepts don't have labels (e.g., \url{Category:South_Korean_social_scientists}), which indicates that the same amount of Wikipedia categories misses natural language descriptions.

\begin{table}
\caption{Analyzed SKOS vocabularies}
\label{tab:vocabs}
\begin{center}
\resizebox{\textwidth}{!} {
\setlength{\extrarowheight}{5pt}

\begin{tabular}{p{6cm}ccccccccc}

\textbf{Vocabulary} & \rotatebox{90}{\textbf{Abbreviation}} & \rotatebox{90}{\textbf{\parbox{2.5cm}{Version/\\Last Modified}}} & \rotatebox{90}{\textbf{Concepts}} & \rotatebox{90}{\textbf{\parbox{2.5cm}{Authoritative\\Concepts}}} & \rotatebox{90}{\textbf{Labels}} & \rotatebox{90}{\textbf{\parbox{2.5cm}{Semantic\\Relations}}} & \rotatebox{90}{\textbf{\parbox{2.53cm}{Concept\\Schemes}}} \\
\toprule
United Nations Agricultural Thesaurus & \textbf{AGROVOC} & 1.3 & 32,035 & 32,035 & 620,629 & 65,934 & 1 \\
\hline
DBpedia Categories & \textbf{DBpedia} & 3.7 & 743,410 & 743,410 & 740,352 & 1,490,316 & 0 \\
\hline
The EU's Multilingual Thesaurus & \textbf{Eurovoc} & 5.0 & 6,797 & 6,797 & 457,788 & 18,491 & 128 \\
\hline
Geonames Ontology & \textbf{Geonames} & 2.2.1 & 671 & 671 & 671 & 0 & 9 \\
\hline
Gemeenschappelijke Thesaurus Audiovisuele Archieven & \textbf{GTAA} & 2010/08/25 & 171,991 & 171,991 & 178,776 & 50,892 & 9 \\
\hline
Integrated Public Sector Vocabulary & \textbf{IPSV} & 2.00 & 4,732 & 3,080 & 7,945 & 13,843 & 3 \\
\hline
Library of Congress Subject Headings & \textbf{LCSH} & 2012/03/29 & 443,164 & 408,009 & 750,219 & 598,134 & 1 \\
\hline
Austrian Armed Forces Thesaurus & \textbf{LVAk} & 0.9 & 13,411 & 13,411 & 17,250 & 16,346 & 0 \\
\hline
Middle Kingdom Tombs of Ancient Egypt Thesaurus & \textbf{Meketre} & 2011/07/07 & 422 & 422 & 569 & 1,698 & 2 \\
\hline
Medical Subject Headings & \textbf{MeSH} & \cite{van2006method} & 24,626 & 24,626 & 150,617 & 38,858 & 0 \\
\hline
North American Industry Classification System & \textbf{NAICS} & 2012 & 4,175 & 2,213 & 0 & 8,684 & 1 \\
\hline
New York Times People & \textbf{NYTP} & 2010/06/22 & 4,979 & 4,979 & 4,979 & 0 & 1 \\
\hline
University of Southampton Pressinfo & \textbf{Pressinfo} & 2011/02/24 & 1,125 & 1,125 & 0 & 0 & 0 \\
\hline
Peroxisome Knowledge Base & \textbf{PXV} & 1.6 & 2,112 & 1,686 & 3,628 & 2,695 & 1 \\
\hline
Thesaurus for Economics & \textbf{STW} & 8.06 & 6,524 & 6,524 & 31,189 & 57,907 & 1 \\
\bottomrule
\end{tabular}
}
\end{center}
\end{table}

\subsection{Results}

The results of this analysis are summarized in Table~\ref{tab:results}, which shows the absolute number of possibly affected resources for each quality checking function and vocabulary combination. n/a means that a certain function was not applicable for reasons we explain below. Numbers marked with an asterisk (*) were obtained by extrapolating from subsets containing 5\% of the respective vocabulary resources.

\begin{table}[h]
\caption{Results of the quality checking functions}
\label{tab:results}

\begin{center}
\resizebox{\textwidth}{!} {
\setlength{\extrarowheight}{5pt}

\begin{tabular}{lccccccccccccccc}
\textbf{Issue} & \rotatebox{90}{\textbf{AGROVOC}} & \rotatebox{90}{\textbf{DBpedia}} & \rotatebox{90}{\textbf{Eurovoc}} & \rotatebox{90}{\textbf{Geonames}} & \rotatebox{90}{\textbf{GTAA}} & \rotatebox{90}{\textbf{IPSV}} & \rotatebox{90}{\textbf{LCSH}} & \rotatebox{90}{\textbf{LVAk}} & \rotatebox{90}{\textbf{Meketre}} & \rotatebox{90}{\textbf{MeSH}} & \rotatebox{90}{\textbf{NAICS}} & \rotatebox{90}{\textbf{NYTP}} & \rotatebox{90}{\textbf{Pressinfo}} & \rotatebox{90}{\textbf{PXV}} & \rotatebox{90}{\textbf{STW}} \\

\toprule
Omitted or Invalid Language Tags & 0 & 0 & n/a & 0 & 0 & 0 & 100,316 & 13,411 & 0 & 23,950 & n/a & 0 & 1,224 & 1,578 & 2 \\

Incomplete Language Coverage & 32,035 & 0 & n/a & 0 & 0 & 0 & 0 & 0 & 420 & 0 & n/a & 0 & 0 & 0 & 6,456 \\

Undocumented Concepts & 32,035 & 743,410 & 5,341 & 0 & 96,850 & 4,551 & 342,848 & 13,411 & 422 & 1,807 & 3,259 & 4,094 & 1,125 & 1,918 & 5,236 \\

Label Conflicts & 2,949 & 0 & n/a & 18 & 12,404 & 0 & n/a & 13 & 4 & 0 & n/a & 0 & 0 & 7 & 5 \\

\midrule

Orphan Concepts & 0 & 77,062 & 7 & 671 & 162,000 & 0 & 180,909 & 21 & 0 & 0 & 0 & 4,979 & 1,125 & 2 & 4 \\

Weakly Connected Components & 4 & 1,506 & 4 & 0 & 621 & 1 & 22,343 & 11 & 5 & 4 & 1 & 0 & 0 & 10 & 1 \\

Cyclic Hierarchical Relations & 0 & 1,132 & 0 & 0 & 0 & 0 & 0 & 5 & 0 & 4 & 0 & 0 & 0 & 0 & 0 \\

Valueless Associative Relations & 282 & 8,839 & 6 & 0 & 9,448 & 253 & 0 & 5 & 0 & 550 & 0 & 0 & 0 & 0 & 5,139 \\

Solely Transitively Related Concepts & 0 & 0 & 2,652 & 0 & 0 & 0 & 0 & 0 & 36 & 0 & 2,189 & 0 & 0 & 0 & 0 \\

Omitted Top Concepts & 0 & 0 & 1 & 9 & 9 & 0 & 1 & 0 & 0 & 0 & 0 & 1 & 0 & 0 & 0 \\

Top Concept Having Broader Concepts & 0 & 0 & 0 & 0 & 0 & 0 & 0 & 0 & 0 & 0 & 0 & 0 & 0 & 1 & 0 \\

\midrule

Missing In-Links & 32,035 & 733,800 & 6,796 & 19 & 171,980* & 3,080 & 408,000* & 13,411 & 422 & 24,625 & 2,213 & 20 & 1,125 & 1,686 & 6,516 \\

Missing Out-Links & 32,035 & 743,410 & 6,797 & 671 & 171,991 & 0 & 344,054 & 13,411 & 273 & 24,626 & 1 & 0 & 1,116 & 1,046 & 6,524 \\

Broken Links & 238 & 0* & 0* & 0 & 0 & 1 & 148800 & 0 & 425 & 1 & 3,169 & 7 & 11 & 163 & 1 \\

Undefined SKOS Resources & 0 & 0 & 0 & 0 & 0 & 1 & 0 & 0 & 0 & 1 & 0 & 0 & 0 & 0 & 0 \\

\bottomrule
\end{tabular}
}
\end{center}
\end{table}

% LABELING AND DOCUMENTATION ISSUES
We found labeling and documentation issues in all analyzed vocabularies.
% Omitted or Invalid Language Tags
The \texttt{MeSH}, \texttt{PXV}, \texttt{Pressinfo}, and \texttt{LVAk} vocabularies omit language tags with their labeling properties, \texttt{LCSH} with the \texttt{skos:note} property. The \texttt{STW} thesaurus doesn't use language tags with 2 \texttt{skos:definition} property instances.
% Incomplete Language Coverage
\texttt{AGROVOC} covers 25 languages in total but no single concept is labeled in all languages, in \texttt{Meketre} all concepts have English but only some of them French labels assigned. The reason for the high number of concepts with incomplete language coverage in \texttt{STW} is that a few concepts have labels with the private, but valid language tag \texttt{x-other}. 
% Undocumented Concepts
\texttt{Geonames}, which is an ontology including a concept scheme of ``feature codes'', is the only vocabulary in our dataset, which has at least one documentation property assigned to all of its concepts. All other vocabularies have a significant number of undocumented concepts.
% Label Conflicts
We also detected possible label issues in half of the vocabularies. \texttt{PXV}, for instance, uses the string ``primary peroxisomal enzyme deficiency'' with two concepts in the same concept scheme, but once with a \texttt{skos:prefLabel} and another time with a \texttt{skos:altLabel} property. An example from AGROVOC is ``sziv\'{a}rv\'{a}nyos pisztr\'{a}ng'', which is used as \texttt{skos:altLabel} with two different concepts.
% Other stuff
\todo{CM}{The \texttt{Eurovoc} and \texttt{NAICS} vocabularies use SKOS-XL for its labels, which is currently not supported by qSKOS. For computing possible label conflicts for the large number of LCSH labels, we first need to improve our similarity algorithm.}


% STRUCTURAL ISSUES
When analyzing the vocabularies for structural issues, we found that certain results can be seen as indicators for the types of vocabularies.
% Orphan Concepts
In the \texttt{Pressinfo}, \texttt{Geonames}, and \texttt{NYTP} vocabulary, all concepts are orphan concepts, which means that these vocabularies are authority files rather than thesauri or taxonomies. This also implies that these vocabularies have no weakly connected components. \texttt{GTAA} is a mixture of name authority file (approx.~162K concepts) and thesaurus (approx.~10K concepts). The 4 orphan concepts in \texttt{STW} are deprecated concepts.

%the 21 \texttt{LVAk} concepts are obviously test data. <- CM: we should skip this, the preflabels of the concepts seem OK, I have no explanations why they are there

% Weakly Connected Components
Three vocabularies (\texttt{IPSW}, \texttt{NAICS}, \texttt{STW}) consist of only one ``giant component'', which is often considered as being the ideal vocabulary structure. All other vocabularies split into several clusters of semantically related concepts, each of which represents a certain subtopic in the vocabularies. \texttt{Eurovoc}, for instance, contains of 4 weakly connected components, containing 4, 5, 6 and 6775 concepts. The three smallest components contain concepts thematically related to ``Africa'', ``educational institutions'' and ``forecasts''. Weakly connected components divide the \texttt{Meketre} vocabularies into, e.g., topics, and concepts reserved for internal use or museums. \texttt{GTAA} consists of 621 weakly connected components with the largest one being a subject thesaurus containing 8413 concepts. Most of the other components contain less than 10 concepts, representing, e.g., locations, person names, and genres. \texttt{PXV} splits into 10 topic-related weakly connected components, such as ``deficiancies'', ``defects'' or ``signals''.

% Cyclic Hierarchical Relations
Hierarchical cycles are not a common issue except in the collaboratively created DBpedia vocabulary, where many concepts have reflexive \texttt{skos:broader} relations.
% Valuless Associative Relations
Valueless associative relations occur in 8 vocabularies, with their total number being relatively low compared to the total number of all semantic relations in the respective vocabularies.
% Solely Transitive Related Concepts
Solely transitive related concepts occur in 3 vocabularies, establishing relations using properties that, according to the SKOS reference documentation, should not be asserted directly. This indicates a possible misinterpretation of the SKOS specification and could result in a loss in recall on hierarchical queries when the vocabulary is queried without RDFS inferencing.
% Omitted Top Concepts
\texttt{GTAA} and \texttt{Geonames} omit top concepts in all concept schemes they define. \texttt{Eurovoc} uses 128 concept schemes but has one without top concept, which simply contains all concepts defined in the vocabulary. Such an ``umbrella concept scheme'' without top concept is also present in \texttt{LCSH} and \texttt{NYTP}.
% Top Concepts Having Broader Concepts
Only the \texttt{PXV} vocabulary is affected by top concepts having broader concepts in its current version. The reason for this is, according to the vocabulary creator, that the broader concept was abandoned but is still available in the triplestore, maybe because of some bug in the vocabulary management software.

% LINKED DATA SPECIFIC ISSUE

The difference between the number of concepts and the number of authoritative concepts in Table~\ref{tab:vocabs} already indicates which vocabularies are linked with other SKOS vocabularies.
% Missing In-Links
However, except \texttt{NYTP} and \texttt{Geonames}, none of the analyzed vocabularies has a significant number of in-links from other web resources. Estimation of these numbers is rather unaccurate since currently we only rely on the Sindice index. 
% Missing Out-Links
Most vocabularies also don't provide links to other vocabularies on the Web. The \texttt{NYTP}, \texttt{IPSV} vocabularies are the two exceptions, which are fully linked to other Web resources. The one concept with missing out-links in \texttt{NAICS} is the object of a \texttt{skos:broaderTransitive} relation. Some vocabularies, e.g., \texttt{STW}, \texttt{AGROVOC} publish mapping information to other vocabularies as separate files, thus the ``main'' vocabulary lacks this data. A missing in-link count equal to the number of authoritative concepts indicates that mapping information is either not available at all or has to be obtained separately.
% Broken Links
Even though we couldn't determine the exact number of broken links because of the large number of links to resolve  (over 400K in \texttt{Eurovoc}, over 500K in \texttt{LCSH}), we found that broken links are a common issue in most vocabularies.
% Undefined SKOS Resources
Undefined SKOS resources seem to be a minor issue, because we could only find two of them in all vocabularies: \texttt{MeSH} introduces \texttt{skos:annotation} and \texttt{IPSV} uses the deprecated \texttt{skos:prefSymbol} property.

%!TEX root = main.tex

\section{Analysis of Existing SKOS Vocabularies}\label{sec:analysis}

To learn about possible quality issues in real-world vocabularies, we implemented the previously described quality checking function in the qSKOS tool and applied each function on a set of existing SKOS vocabularies.

\subsection{The qSKOS Quality Assessment Tool}

The open-source qSKOS\footnote{\url{https://github.com/cmader/qSKOS/}} quality assessment tool can be used to find possible quality issues in a given SKOS vocabulary. Users can run the tool by passing a vocabulary and selecting the quality checking functions they want to perform. As a result, they obtain detailed reports listing possibly affected resources alongside with the potential cause of the issue. qSKOS is implemented in Java and can be used as standalone command-line tool or API in any other application. Its design is open for the introduction of additional quality checking functions. In our analysis, we used qSKOS version 0.1., which can be downloaded from: \url{https://github.com/cmader/qSKOS/zipball/v0.1}.

\subsection{Vocabulary Data Set}

We used Sindice and available listings\footnote{\url{http://www.w3.org/2001/sw/wiki/SKOS/Datasets}} to learn about existing SKOS vocabularies and selected a subset, which differs in vocabulary size and application domain. We obtained the vocabularies either via direct Web download or by contacting the vocabulary maintainers. Further details about our vocabulary dataset are available in zip-file, which can be downloaded from: \url{http://www.github.com/cmader/???}. It also contains all but the non-public LVAk and PXV vocabularies.

Table~\ref{tab:vocabs} shows our selection and summarizes basic statistical properties relevant for interpreting our results: the number of concepts, ``authoritative concepts'', as explained in Section~\ref{subsec:ld_issues}, all \texttt{skos:prefLabel}, \texttt{skos:altLabel}, and \texttt{skos:hiddenLabel} relations (\texttt{Labels}), all asserted semantic relations $SR$, as well as the number of concept schemes.


\begin{table}
\caption{Analyzed SKOS vocabularies}
\label{tab:vocabs}
  
\begin{center}
\resizebox{\textwidth}{!} {
\setlength{\extrarowheight}{5pt}

\begin{tabular}{p{6cm}ccccccccc}

\textbf{Vocabulary} & \rotatebox{90}{\textbf{Abbreviation}} & \rotatebox{90}{\textbf{\parbox{3cm}{Version/\\Last Modified}}} & \rotatebox{90}{\textbf{Concepts}} & \rotatebox{90}{\textbf{Auth. Concepts}} & \rotatebox{90}{\textbf{Labels}} & \rotatebox{90}{\textbf{Semantic Rel.}} & \rotatebox{90}{\textbf{Concept Schemes}} \\
\toprule
Agricultural Thesaurus & \textbf{AGROVOC} & 1.3 & 32,035 & 32,035 & 620,629 & 65,934 & 1 \\
\hline
DBpedia Categories & \textbf{DBpedia} & 3.7 & 743,410 & 743,410 & 740,352 & 1,490,316 & 0 \\
\hline
The EU's multilingual thesaurus & \textbf{Eurovoc} & 5.0 & 6,797 & 6,797 & 457,788 & 18,491 & 128 \\
\hline
Geonames Ontology & \textbf{Geonames} & 2.2.1 & 671 & 671 & 671 & 0 & 9 \\
\hline
Gemeenschappelijke Thesaurus Audiovisuele Archieven & \textbf{GTAA} & 2010/08/25 & 171,991 & 171,991 & 178,776 & 50,892 & 9 \\
\hline
Integrated Public Sector Vocabulary & \textbf{IPSV} & 2.00 & 4,732 & 3,080 & 7,945 & 13,843 & 3 \\
\hline
Library of Congress Subject Headings & \textbf{LCSH} & 2011/08/09 & 459,182 & 407,908 & 746,076 & 595,754 & 19 \\
\hline
Austrian Armed Forces Thesaurus & \textbf{LVAk} & 0.9 & 13,411 & 13,411 & 17,250 & 16,346 & 0 \\
\hline
Middle Kingdom tombs of Ancient Egypt Thesaurus & \textbf{Meketre} & 2011/07/07 & 422 & 422 & 569 & 1,698 & 2 \\
\hline
Medical Subject Headings & \textbf{MeSH} & \cite{van2006method} & 24,626 & 24,626 & 150,617 & 38,858 & 0 \\
\hline
North American Industry Classification System & \textbf{NAICS} & 2012 & 4,175 & 2,213 & 0 & 8,684 & 1 \\
\hline
New York Times People & \textbf{NYTP} & 2010/06/22 & 4,979 & 4,979 & 4,979 & 0 & 1 \\
\hline
University of Southampton Pressinfo & \textbf{Pressinfo} & 2011/02/24 & 1,125 & 1,125 & 0 & 0 & 0 \\
\hline
Peroxisome Knowledge Base & \textbf{PXV} & 1.6 & 2,112 & 1,686 & 3,628 & 2,695 & 1 \\
\hline
Thesaurus for Economics & \textbf{STW} & 8.06 & 6,524 & 6,524 & 31,189 & 57,907 & 1 \\
\bottomrule
\end{tabular}

}
\end{center}
\end{table}

\subsection{Results}

The results of this analysis are summarized in Table~\ref{tab:results}, which shows the absolute number of affected resources for each quality checking function and vocabulary combination. The value 0 means that we haven't found any affected resource, n/a means that a certain function was not applicable for reasons we explain below. Numbers marked with an asterisk (*) were obtained by extrapolating from subsets containing 5\% of the respective vocabulary resources.

\begin{table}[h]
\caption{Results of the quality checking functions}
\label{tab:results}

\begin{center}
\resizebox{\textwidth}{!} {
\setlength{\extrarowheight}{5pt}

\begin{tabular}{lccccccccccccccc}
\textbf{Issue} & \rotatebox{90}{\textbf{AGROVOC}} & \rotatebox{90}{\textbf{DBpedia}} & \rotatebox{90}{\textbf{Eurovoc}} & \rotatebox{90}{\textbf{Geonames}} & \rotatebox{90}{\textbf{GTAA}} & \rotatebox{90}{\textbf{IPSV}} & \rotatebox{90}{\textbf{LCSH}} & \rotatebox{90}{\textbf{LVAk}} & \rotatebox{90}{\textbf{Meketre}} & \rotatebox{90}{\textbf{MeSH}} & \rotatebox{90}{\textbf{NAICS}} & \rotatebox{90}{\textbf{NYTP}} & \rotatebox{90}{\textbf{Pressinfo}} & \rotatebox{90}{\textbf{PXV}} & \rotatebox{90}{\textbf{STW}} \\

\toprule
Omitted or Invalid Language Tags & 0 & 0 & n/a & 0 & 0 & 0 & 18 & 13,411 & 0 & 23,950 & n/a & 0 & 1,224 & 1,578 & 2 \\

Incomplete Language Coverage & 32,035 & 0 & n/a & 0 & 0 & 0 & 0 & 0 & 420 & 0 & n/a & 0 & 0 & 0 & 6,456 \\

Undocumented Concepts & 32,035 & 743,410 & 5,341 & 0 & 96,850 & 4,551 & 398,036 & 13,411 & 422 & 1,807 & 3,259 & 4,094 & 1,125 & 1,918 & 5,236 \\

Label Conflicts & 2,949 & 0 & n/a & 18 & 12,404 & 0 & n/a & 13 & 4 & 0 & n/a & 0 & 0 & 7 & 5 \\

\midrule

Orphan Concepts & 0 & 77,062 & 7 & 671 & 162,000 & 0 & 172,364 & 21 & 0 & 0 & 0 & 4,979 & 1,125 & 2 & 4 \\

Weakly Connected Components & 4 & 1,506 & 4 & 0 & 621 & 1 & 22,131 & 11 & 5 & 4 & 1 & 0 & 0 & 10 & 1 \\

Cyclic Hierarchical Relations & 0 & 1,132 & 0 & 0 & 0 & 0 & 0 & 5 & 0 & 4 & 0 & 0 & 0 & 0 & 0 \\

Valueless Associative Relations & 282 & 8,120 & 1 & 0 & 9,438 & 239 & 1,879 & 5 & 0 & 495 & 0 & 0 & 0 & 0 & 5,082 \\

Solely Transitively Related Concepts & 0 & 0 & 2,652 & 0 & 0 & 0 & 0 & 0 & 36 & 0 & 2,189 & 0 & 0 & 0 & 0 \\

Omitted Top Concepts & 0 & 0 & 1 & 9 & 9 & 0 & 19 & 0 & 0 & 0 & 0 & 1 & 0 & 0 & 0 \\

Top Concept Having Broader Concepts & 0 & 0 & 0 & 0 & 0 & 0 & 0 & 0 & 0 & 0 & 0 & 0 & 0 & 1 & 0 \\

\midrule

Missing In-Links & 32,035 & 733,800 & 6,796 & 19 & 171,980* & 3,080 & 404,540* & 13,411 & 422 & 24,625 & 2,213 & 20 & 1,125 & 1,686 & 6,516 \\

Missing Out-Links & 32,035 & 743,410 & 6,797 & 671 & 171,991 & 0 & 408,198 & 13,411 & 273 & 24,625 & 0 & 0 & 1,116 & 1,472 & 6,524 \\

Broken Links & 238 & 0* & 0* & 0 & 0 & 1 && n/a & 425 & 1 & 3,169 & 7 & 11 & 163 & 1 \\

Undefined SKOS Resources & 0 & 0 & 0 & 0 & 0 & 1 & 0 & 0 & 0 & 1 & 0 & 0 & 0 & 0 & 0 \\

\bottomrule
\end{tabular}
}
\end{center}
\end{table}

% LABELING AND DOCUMENTATION ISSUES
We found labeling and documentation issues in all analyzed vocabularies. 
% Omitted or Invalid Language Tags
The \texttt{MeSH}, \texttt{PXV}, \texttt{Pressinfo}, and \texttt{LVAk} vocabularies omit language tags with their \texttt{skos:prefLabel}, \texttt{skos:altLabel}, or \texttt{skos:hiddenLabel} properties. The STW thesaurus doesn't use language tags with 2 \texttt{skos:definition} property instances and the LCSH vocabulary uses language tags with concepts, but not with concept schemes.
% Incomplete Language Coverage
\texttt{AGROVOC} covers 25 languages in total but no single concept is labelled in all languages, in \texttt{Meketre} all concepts have English but only some of them French labels assigned. The reason for the high number of concepts with incomplete language coverage in \texttt{STW} is that a few concepts have labels with the private, but valid language tag \texttt{x-other}. 
% Undocumented Concepts
\texttt{Geonames}, which is an ontology including a concept scheme of ``feature codes'', is the only vocabulary in our dataset, which has at least one documentation property assigned to all of its concept. All other vocabularies have a significant number of undocumented concepts.
% Label Conflicts
We also detected possible label conflicts in half of the vocabularies. \texttt{PXV}, for instance, uses the string ``primary peroxisomal enzyme deficiency'' with two concepts in the same concept scheme, but once with a \texttt{skos:prefLabel} and another time with a \texttt{skos:altLabel} property. A similar example from AGROVOC is ``sziv\'{a}rv\'{a}nyos pisztr\'{a}ng'', which is used as \texttt{skos:altLabel} with two different concepts.
% Other stuff
The \texttt{Eurovoc} and \texttt{NAICS} vocabularies use SKOS-XL for its labels, which is currently not supported by qSKOS.


% STRUCTURAL ISSUES
When analyzing the structure of existing vocabularies, we found that the results of certain quality checks can be seen as indicators for the types of vocabularies.
% Orphan Concepts
In the \texttt{Pressinfo} and \texttt{NYTP} vocabulary, all concepts are orphan concepts, which means that these vocabularies are authority files rather than thesauri or taxonomies. \texttt{GTAA} seems is a mixture of name authority file (162K concepts) and thesaurus (10K concepts). The 4 orphan concepts in \texttt{STW} are deprecated concepts.

%the 21 \texttt{LVAk} concepts are obviously test data. <- CM: we should skip this, the preflabels of the concepts seem OK, I have no explanations why they are there

% Weakly Connected Components
Consisting of one weakly connected component (``giant component''), three vocabularies represent what is often considered as the ideal vocabulary structure. All other vocabularies split into several clusters of semantically related concepts that represent several subtopics in the vocabularies. \texttt{Eurovoc} contains of 4 weakly connected components, containing 4, 5, 6 and 6775 concepts. The three smallest components contain concepts thematically related to ``Africa'', ``educational institutions'' and ``forecasts''. Weakly connected components divide the \texttt{Meketre} vocabularies into, e.g., execution-style related concepts, concepts reserved for internal use or museums. \texttt{GTAA} consists of 621 weakly connected components with the largest one supposed to be a subject thesaurus containing 8413 concepts. Most  other components contain less than 10 concepts, representing, e.g., locations, person names, maker names and genres. \texttt{PXV} splits into 10 weakly connecting components, topically relating e.g., ``deficiancies'', ``defects'' or ``signals''.
% Cyclic Hierarchical Relations
Hierarchical cycles are not a common issue except in collaboratively created DBpedia vocabulary, where many concepts have reflexive \texttt{skos:broader} relations.
% Valuless Associative Relations
Valueless associative relations occur in 9 vocabularies, with their total number being relatively low compared to the total number of all semantic relations in the respective vocabularies.
% Solely Transitive Related Concepts
Solely transitive related concepts occur in 3 vocabularies, establishing relations using properties that, according to the SKOS reference documentation, should not be asserted directly. This indicates a possible misunderstanding of the SKOS specification and, when the vocabulary is queried without RDFS inferencing, could result in a loss in recall on hierarchical queries.
% Omitted Top Concepts
\texttt{GTAA}, \texttt{Geonames}, \texttt{NYTP}, and \texttt{LCSH} have no top concept defined for any of the concept schemes they use. \texttt{Eurovoc} uses 128 concept schemes but has one without top concept, which simply contains all concepts defined in the vocabulary.
% Top Concepts Having Broader Concepts
Only \texttt{PXV} vocabulary is affected by top concepts having broader concepts in its current version. The reason for this is, according to the vocabulary creator, that the broader concept was abandoned but is still available in the triplestore, maybe because of some bug in the vocabulary management software.

% LINKED DATA SPECIFIC ISSUE

The difference between Concepts and Authoritative Concepts in Table~\ref{tab:vocabs} already indicates which vocabularies are linked with SKOS vocabularies.
% Missing In-Links
However, except \texttt{NYTP} and \texttt{Geonames}, none of the analyzed vocabularies has a significant number of in-links from other web resources.
% Missing Out-Links
Most vocabularies also don't provide links to other vocabularies on the Web. The \texttt{NYTP}, \texttt{IPSV} and \texttt{NAICS} vocabularies are the three exceptions, which are fully linked to other Web resources. 
% Broken Links
Even though we couldn't determine the exact number of broken links because of the large number of links to resolve  (over 400K in \texttt{Eurovoc} and \texttt{LCSH}), we found that broken links are a common issue in most vocabularies.
% Undefined SKOS Resources
Undefined SKOS resources seem to be a minor issue, because we could only find two of them in all vocabularies: \texttt{MeSH} introduces \texttt{skos:annotation} and \texttt{IPSV} uses the deprecated \texttt{skos:prefSymbol} property.

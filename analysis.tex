%!TEX root = main.tex

\section{Analysis of Existing SKOS Vocabularies}\label{sec:analysis}

To learn about possible quality issues in real-world vocabularies, we implemented the previously described quality checking function in the qSKOS tool and applied each function on a set of existing SKOS vocabularies. 

\subsection{The qSKOS Quality Checking Tool}

The open-source qSKOS\footnote{\url{https://github.com/cmader/qSKOS/}} quality assessment tool can be used to find possible quality issues a given SKOS vocabulary. Users can run the tool by passing a vocabulary and selecting the quality checking functions they want to perform. As a result, they obtain detailed reports listing possibly affected resources alongside with the potential cause of the issue. qSKOS is implemented in Java and can be used as standalone command-line tool or API in any other application. Its design is open for the introduction of additional quality functions. In our analysis, we used qSKOS version 0.2., which can be downloaded from: \url{https://github.com/cmader/qSKOS/releases/qSKOS-0.2.tar.gz}.

\subsection{Vocabulary Data Set}

We used Sindice and available listings\footnote{\url{http://www.w3.org/2001/sw/wiki/SKOS/Datasets}} to learn about existing SKOS vocabularies. Table~\ref{tab:vocabs} lists our representative vocabulary selection and shows that the vocabularies differ in size and application domain: the \emph{Gemeenschappelijke Thesaurus Audiovisuele Archieven (GTAA)} is vocabulary from the media domain, \emph{Eurovoc} and the \emph{Integrated Public Sector Vocabularies (PSV)} cover public sector concepts, the \emph{Medical Subject Headings (MeSH)} and \emph{Peroxisome Knowledge Base (PXV)} are from the (bio-)medical domain, the \emph{Geonames Ontology} covers locations, the \emph{Thesaurus for Economics (STW)} and \emph{North American Industry Classification System (NAICS)} are vocabularies from the economics and business area and the \emph{DBPedia Categories} reflect Wikipedia's user-generated categorization system. We also include the \emph{Meketre} vocabulary, which defines Egyptology-related concepts, and the non-public \emph{Austrian Armed Forces Thesaurus (LVAk)}, which we generated from CSV files provided by the vocabulary maintainers. A zip-file containing all 14 public vocabulariesx we used in our analysis, can be downloaded from: \url{http://www.github.com/cmader/???}.

\begin{table}
\label{tab:vocabs}
\caption{Analyzed SKOS vocabularies}
    
\begin{center}
\resizebox{\textwidth}{!} {
\setlength{\extrarowheight}{5pt}

\begin{tabular}{p{6cm}ccccccccc}

\textbf{Vocabulary} & \rotatebox{90}{\textbf{Abbreviation}} & \rotatebox{90}{\textbf{Version/last mod.}} & \rotatebox{90}{\textbf{Concepts}} & \rotatebox{90}{\textbf{Auth. Concepts}} & \rotatebox{90}{\textbf{Labels}} & \rotatebox{90}{\textbf{Semantic Rel.}} & \rotatebox{90}{\textbf{Aggregation Rel.}} & \rotatebox{90}{\textbf{Concept Schemes}} \\
\toprule
Agricultural Thesaurus & \textbf{AGROVOC} & 1.3 & 32,035 & 32,035 & 620,629 & 65,934 & 32,085 & 1 \\
\hline
DBpedia Categories & \textbf{DBpedia} & ? & 743,410 & 743,410 & 740,352 & 1,490,316 & 0 & 0 \\
\hline
The EU's multilingual thesaurus & \textbf{Eurovoc} & 5.0 & 6,797 & 6,797 & 457,788 & 18,491 & 15,512 & 128 \\
\hline
Geonames Ontology & \textbf{Geonames} & 2.2.1 & 671 & 671 & 671 & 0 & 671 & 9 \\
\hline
Gemeenschappelijke Thesaurus Audiovisuele Archieven & \textbf{GTAA} & 2010/08/25 & 171,991 & 171,991 & 178,776 & 50,892 & 343,980 & 9 \\
\hline
Integrated Public Sector Vocabulary & \textbf{IPSV} & 2.00 & 4,732 & 3,080 & 7,945 & 13,843 & 4,483 & 3 \\
\hline
Library of Congress Subject Headings & \textbf{LCSH} & 2011/08/09 & 459,182 & 407,908 & 746,076 & 595,754 & 815,816 & 19 \\
\hline
Austrian Armed Forces Thesaurus & \textbf{LVAk} & n/a & 13,411 & 13,411 & 17,250 & 16,346 & 0 & 0 \\
\hline
Middle Kingdom tombs of Ancient Egypt Thesaurus & \textbf{Meketre} & 2011/07/07 & 422 & 422 & 569 & 1,698 & 6 & 2 \\
\hline
Medical Subject Headings & \textbf{MeSH} & ? & 24,626 & 24,626 & 150,617 & 38,858 & 0 & 0 \\
\hline
North American Industry Classification System & \textbf{NAICS} & 2012 & 4,175 & 2,213 & 0 & 8,684 & 2,235 & 1 \\
\hline
New York Times People & \textbf{NYTP} & 2010/06/22 & 4,979 & 4,979 & 4,979 & 0 & 4,979 & 1 \\
\hline
University of Southampton Pressinfo & \textbf{Pressinfo} & 2011/02/24 & 1,125 & 1,125 & 0 & 0 & 0 & 0 \\
\hline
Peroxisome Knowledge Base & \textbf{PXV} & 1.6 & 2,112 & 1,686 & 3,628 & 2,695 & 1,716 & 1 \\
\hline
Thesaurus for Economics & \textbf{STW} & 8.06 & 6,524 & 6,524 & 31,189 & 57,907 & 6,531 & 1 \\
\bottomrule
\end{tabular}

}
\end{center}
\end{table}

\subsection{Results}

The results of this analysis are reported in Table~\ref{tab:results}, which shows the absolute number of possibly affected resources for each quality checking function and vocabulary combination. The value 0 means that we haven't found any affected resource, N/A means that a certain function was not applicable for reasons we explain below. An asterisk after certain numeric values indicates that due to performance reason, only an extrapolated subset containing 5\% of the respective elements (HTTP URIs or concepts) of the vocabulary has been analyzed.

\begin{table}[h]
\label{tab:results}
\caption{Results of the quality checking functions}

\begin{center}
\resizebox{\textwidth}{!} {
\setlength{\extrarowheight}{5pt}

\begin{tabular}{p{4cm}ccccccccccccccc}
\textbf{Issue} & \rotatebox{90}{\textbf{GTAA}} & \rotatebox{90}{\textbf{Geonames}} & \rotatebox{90}{\textbf{MeSH}} & \rotatebox{90}{\textbf{PXV}} & \rotatebox{90}{\textbf{Eurovoc}} & \rotatebox{90}{\textbf{IPSV}} & \rotatebox{90}{\textbf{Agrovoc}} & \rotatebox{90}{\textbf{DBpedia}} & \rotatebox{90}{\textbf{Pressinfo}} & \rotatebox{90}{\textbf{NYTP}} & \rotatebox{90}{\textbf{LCSH}} & \rotatebox{90}{\textbf{Meketre}} & \rotatebox{90}{\textbf{STW}} & \rotatebox{90}{\textbf{NAICS}} & \rotatebox{90}{\textbf{LVAk}} \\
\toprule
Omitted or Invalid Language Tags & 0 & 0 & 23,950 & 1,578 & n/a & 0 & 0 & 0 & 1,224 & 0 & 18 & 0 & 2 & n/a & 13,411 \\

Incomplete Language Coverage & 0 & 0 & 0 & 0 & n/a & 0 & 32,035 & 0 & 0 & 0 & 0 & 420 & 6,456 & n/a & 0 \\

Undocumented Concepts & 96,850 & 0 & 1,807 & 1,918 & 5,341 & 4,551 & 32,035 & 743,410 & 1,125 & 4,094 & 398,036 & 422 & 5,236 & 3,259 & 13,411 \\

Label Conflicts & 12,404 & 18 & 0 & 7 & n/a & 0 & 2,949 & 0 & 0 & 0 & tba & 4 & 5 & n/a & 13 \\

\midrule

Orphan Concepts & 162,000 & 671 & 0 & 2 & 7 & 0 & 0 & 77,062 & 1,125 & 4,979 & 172,364 & 0 & 4 & 0 & 21 \\

Weakly Connected Components & 621 & 0 & 4 & 10 & 4 & 1 & 4 & 1,506 & 0 & 0 & 22,131 & 5 & 1 & 1 & 11 \\

Cyclic Hierarchical Relations & 0 & 0 & 4 & 0 & 0 & 0 & 0 & 1,132 & 0 & 0 & 0 & 0 & 0 & 0 & 5 \\

Valueless Associative Relations & 9,438 & 0 & 495 & 0 & 1 & 239 & 282 & 8,120 & 0 & 0 & 1,879 & 0 & 5,082 & 0 & 5 \\

Solely Transitively Related Concepts & 0 & 0 & 0 & 0 & 2,652 & 0 & 0 & 0 & 0 & 0 & 0 & 36 & 0 & 2,189 & 0 \\

Omitted Top Concepts & 9 & 9 & 0 & 0 & 1 & 0 & 0 & 0 & 0 & 1 & 18 & 0 & 0 & 0 & 0 \\

Top Concept Having Broader Concepts & 0 & 0 & 0 & 1 & 0 &0 & 0 & 0 & 0 & 0 & 0 & 0 & 0 & 0 & 0 \\

\midrule

Missing In-Links & 171,980* & 19 && 1,686 & 6,796 & 3,080 & 32,035 & 733,800* & 1,125 & 20 & 404,540* & 422 & 6,516 & 2,213 & 13,411 \\

Missing Out-Links & 171,991 & 671 && 1,472 & 6,797 & 0 & 32,035 & 743,410 & 1,116 & 0 & 408,198 & 273 & 6,524 & 0 & 13,411 \\

Broken Links & 0 & 0 & 1 & 163 &&& 238 & 0* & 11 & 7 && 425 & 1 & 3,169 & n/a \\


Undefined SKOS Resources & 0 & 0 & 1 & 0 & 0 & 1 & 0 & 0 & 0 & 0 & 0 & 0 & 0 & 0 & 0  \\

\bottomrule
\end{tabular}
}
\end{center}
\end{table}

% Labeling and Documentation Issues
\todo{BH}{this sentence is not clear, I assume you mean language tags, not labels.I'm not sure if relating language tags to concepts makes sense because a concept can be connected to multiple labels}
We found labeling and documentation issues in all analyzed vocabularies. PXV provides labels for XX\% of its concepts, Pressinfo labels only for 1 out of 1125 concepts, and the STW thesaurus misses labels for two concepts. Eurovoc and NAICS use SKOS-XL labels, which are currently not supported by qSKOS.

Incomplete Language Coverage were issues in the three analyzed vocabularies: in AGROVOC all concepts had incomplete coverage, probably because \todo{CM}{es gibt labels in insgesamt 25 versch sprachen und keines der konzepte ist in jeder sprache beschrieben}, and in STW we found XX\%.

Most vocabularies didn't document their concepts with the documentary properties defined by SKOS.

Label conflicts, hence \todo{CM}{explain briefly} were found in several vocabularies: GTAA has many because? PXV has 7. Why given examples?

% Structural Issues
orphan concepts: seem to depend on vocab use case: list of names: many orphans (gtaa,lcsh,geonames); vocabs establishing a network of terms: less to none orphan concepts

wcc: "islands" of semantically related concepts; most interesting cases: vocabs having few orphan concepts but more than one wcc: vocab may split into clusters inintentionally

cycles: not a common issue, dbpedia category cylces were somewhat surprising

valueless associative relations: in 9 of 15 vocabs; comparatively few in relation to all semantic relations

solely transitively related concepts: only 3 of 15 vocabs

omitted top concepts: 11 vocabs have at least one concept scheme defined, 5 have missing top concepts; gtaa,geonames,nyt people have no top concepts for any concept scheme, eurovoc uses 128 concept schemes and only one without top concept; opposite case for lcsh: uses 19 concept schemes defined, only one has a top concept

top concepts having broader concept: only pxv vocab affected in current version: in earlier version a different concept featuring this issue could be spotted: according to vocab creator this was an erroneous concept still available in the triplestore; maybe introduced by some bug in the vocab management software

% Linked Data Specific Issues
missing in-links: authoritative conceps of most vocabs aren't externally linked; exception: nyt people vocab

missing out-links: many vocab's also don't provide links to other vocabs; exception: naics, nyt people

broken loinks: meketre: reason server-side misconfiguration (fixed in current version); dbpedia sometimes down, link checking for some vocabs difficult: our algorithm waits 3secs when dereferencing links to not produce an uninteded DOS attack => estimation was necessary

undef skos resources: seems to be a minor problems, the examined voabs seem to be well-maintained 


% The \textbf{Loose Concepts} metrics provides insights on the structural complexity of vocabularies. The large number of loose concepts in GTAA can be explained by the fact that it contains names (e.g., persons or places) that arent't connected using skos properties. This is also the case for Geonames, NYTP and Pressinfo where the number of total conceps is equal to the number of loose concepts, indicating that no concept is semantically related to another concept.
% 
% The large number of \textbf{Weakly Connected Components} in GTAA is caused by many ``minimal'' components, containing only 2 concepts connected by a \texttt{skos:related} property. Due to the fact that loose concepts are not counted as weakly connected components, Geonames, NYTP and Pressinfo evaluate to zero components.

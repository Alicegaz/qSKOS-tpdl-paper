%!TEX root = main.tex
\section{Related Work}\label{sec:related_work}

% Data quality in general

The problem of ``vocabulary quality'' is closely related to the more general problem of ``data quality`'', which has intensively been discussed in data and information systems research (cf.~\cite{Batini2009}). Pipino et al.~\cite{Pipino2002} argue that dealing with data quality should involve both ``subjective perceptions of the individuals'' and ``objective measurements based on the data set in question''. We see our work as a contribution to the latter and believe that the kind of quality assessment we support must be combined with domain knowledge, and therefore human expertise, to make final quality judgements.

% Quality in Linked Data

Data quality research has also been conducted in the Semantic Web and Linked Data research areas. Hogan et. al~\cite{Hogan2010} identify four categories of common errors and shortcomings in RDF documents and also Heath and Bizer~\cite{Heath2011} summarize best practices for publishing data on the Web. However, the authors focus on RDF datasets in general without considering SKOS-specific properties. 

% Quality in ontology engineering

Ontology evaluation is task of measuring the quality of an ontology and has extensively been discussed by Vrandecic~[\todo{CM}{ref denny}]. Gangemi [\todo{CM}{ref ?}] introduced structural, functional and usability-related measures for ontology evaluation, Arpinar [\todo{CM}{ref ?}] define three types of conflicts that may occur in ontologies, and also Tartir et al.[\todo{CM}{ref ?}] propose a framework for evaluation and ranking of ontologies. Although there are definitely overlaps between ontology evaluation and our research, the quality issues we identified are less tight to underlying data model and formal ontology definition language but, to a large extent, derived from manuals and guidelines taxonomist use in their daily work.
\documentclass{llncs}

\usepackage{verbatim}
\usepackage{url}
\usepackage{amssymb}
\usepackage{rotating}
\usepackage{color}
\usepackage{booktabs}

\newcommand{\todo}[2]{\textbf{\textcolor{red}{(TODO [#1]: #2)}}}

% metrics = quantitative indicator of sth

\title{Quality Metrics for SKOS Vocabularies}
\author{Christian Mader\inst{1} \and Bernhard Haslhofer\inst{2}}
\institute{
	University of Vienna, Faculty of Computer Science\\\email{christian.mader@univie.ac.at}
	\and Cornell University, Information Science\\\email{bernhard.haslhofer@cornell.edu}}

\begin{document}

\maketitle

\begin{abstract}
    
The Simple Knowledge Organization System (SKOS) has become the standard model for expressing and publishing controlled vocabularies on the Web. However, SKOS vocabularies are often heterogeneous in terms of quality, which reduces their applicability across system boundaries. Here we investigate how we can support domain experts with quantitative indicators to judge the quality of SKOS vocabularies. Based on existing guidelines and best practices for creating controlled vocabularies, we identified existing \textbf{quality criteria} and formalized them into computable \textbf{quality metrics}. We implemented them in the qSKOS quality assessment tool and used it to analyze the quality of existing vocabularies. The results from this analysis indicate that \todo{CM,BH}{summarize results in one sentence}.

\end{abstract}

%!TEX root = main.tex

\section{Introduction}\label{sec:introduction}

% BACKGROUND

% What is SKOS?
The Simple Knowledge Organization System (SKOS)~\cite{SkosReference2008} is a standard model for sharing and linking controlled vocabularies (thesauri, classification systems, etc.) on the Web. Many organizations, including the European Union\footnote{EuroVoc, \url{http://eurovoc.europa.eu/}}, the United Nations\footnote{AGROVOC, \url{http://aims.fao.org/website/AGROVOC-Thesaurus/sub}}, or the UK government\footnote{Integrated Public Sector Vocabulary (IPSV), \url{http://doc.esd.org.uk/IPSV}} publish SKOS representations of their vocabularies on the Web so that they can easily be accessed by humans and machines.

However, quality issues can affect the applicability of SKOS vocabularies for tasks such as query expansion, faceted browsing, or auto-completion, as in the following examples: 

\begin{itemize}

	\item AGROVOC defines concepts in 25 different languages. However, while most concepts have English labels attached, only 38\% have German labels. This can be a problem for multilingual applications that rely on label translations.

	\item An earlier version of the STW thesaurus contained 5 pairs of concepts with identical labels. As a result, the auto-complete function of the online search interface suggested identical entries without disambiguation information.

	\item The non-public vocabulary of the Austrian Armed Forces (LVAk) contains 11 disconnected concept clusters. When confronted with these structures, the thesaurus maintainers recognized them as ``forgotten'' test data that has no practical significance.
	
\end{itemize}

% PROBLEM / RESEARCH QUESTION

The SKOS specification defines a set of \textbf{integrity conditions} that state whether given data patterns are consistent with the SKOS model. Yet the SKOS integrity conditions fail to capture quality aspects like the ones above. The main reason lies in SKOS' ``minimal commitment'' approach. A standard that aims at cross-domain interoperability should refrain from defining constraints that impose on one domain the requirements of another. SKOS is thus very liberal with respect to data integrity.

On the other hand, each vocabulary should fulfill domain- and application-specific quality aspects and taxonomists often follow standard guidelines specific to given types of vocabularies (cf.,~\cite{ISO25964-1:2011,Z39.19:2005}) for thesauri) or apply their own hand-crafted checks~\cite{Coronado2009}. Existing guidelines consider these aspects, but currently rely on \textbf{human judgement}, which is subjective and does not scale for larger vocabularies. The SKOS context, where vocabularies can be linked together on the Web, also brings issues hitherto unforeseen by traditional checking approaches.

% CONTRIBUTION

We aim at contributing to the ongoing community efforts to bridge that gap between model-level integrity constraints and domain-specific quality aspects. Our goal is to help taxonomists in identifying possible quality issues in SKOS vocabularies and to give them a set of computable quality checking functions that, in combination with the taxonomists' experience and domain expertise, can serve as quality indicators for vocabularies. Finding such quality issues also gives important feedback on the overall vocabulary design process and should, at the end, lead to better vocabularies. The main contributions of this work can be summarized as follows:

\begin{itemize}

	\item We identified 15 \textbf{quality issues} for SKOS vocabularies by examining existing guidelines and formalized them into computable \textbf{quality checking functions} that identify possibly affected resources in a vocabulary.
	
	\item With the \textbf{qSKOS quality assessment tool} we provide a reference implementation of these functions.

	\item We \textbf{tested} these functions by \textbf{analyzing} a representative set of \textbf{15 existing SKOS vocabularies} to learn about possible quality issues.

\end{itemize}

In the following, we will first discuss what \emph{quality} means in the context of SKOS vocabularies and how it is currently support by the SKOS specification and existing tools. Then we introduce the quality issues we have identified and describe how we implemented them in the qSKOS quality assessment tool. Finally, we report on the results of an analysis we performed on 15 existing SKOS vocabularies and which shows that the quality issues we discussed are real and can lead to the improvement of existing vocabularies.

%!TEX root = main.tex

\section{Background and Methodology}

% About Quality in SKOS vocabularies

SKOS is a language for defining vocabularies in the Web of Data and therefore based on the Open World Assumption. Established quality notions from closed-world systems, such as referential integrity or schema validation, don’t hold anymore, because the available information may be incomplete and facts that are not explicitly stated cannot be determined as true or false. While trust and provenance models for Web data are being developed~\cite{Omitola2011,Hartig2009}, content-based and hand-crafted heuristics are currently used to evaluate quality in Linked Data sets~\cite{Heath2011}.

The quality of a given SKOS vocabulary can have a direct impact on the following application areas of controlled vocabularies: \todo{CM}{write a paragraph}

SKOS consistency and integrity means that
- 


\todo{CM}{After the introduction it is very common to introduce the problem in more detail. We should write a page about quality in controlled vocabularies and explain the sources for our quality metics. Maybe you can pull parts from the related work section.}

% Methodology

\todo{CM}{How did we identify the criteria we present in the following section?}



%!TEX root = main.tex

\section{Quality Metrics for SKOS Vocabularies}\label{sec:metrics}

In the following we describe the design rationale and the semi-formal definitions for the \textbf{quality metrics}, which we derived from the quality criteria we have found. These metrics can be implemented on top of the SKOS model, which we define as follows:

% This definition is not a quality criterions, henec it is in the background section (we need to discuss that)

\begin{definition}[SKOS Data Model] Let R be the set of all resources, as defined in \cite{Jacobs2004} and $V = (C, A, LL, LR, SR, AR)$ be a SKOS vocabulary, with

\begin{itemize}
	\item \(C \subseteq R\) being the set of \textbf{conceptual resources} of type \texttt{skos:Concept}.
	\item \(A \subseteq R\) being the set of aggregation and grouping resources in SKOS, hence \texttt{skos:ConceptScheme} and \texttt{skos:Collection}.
	\item LL being the set of \textbf{lexical labels}, which are instances of RDF plain literals.
	\item \(LR \subseteq C \times LL\) being the set of \textbf{lexical relations} associating conceptual resources with lexical labels, hence instances of \texttt{skos:prefLabel}, \texttt{skos:altLabel}, or \texttt{skos:hiddenLabel}.
\item \(SR \subseteq C \times C\) being the set of \textbf{semantic relations} associating conceptual resources with concepts, hence instances of \texttt{skos:semanticRelation} and subproperties thereof.
\item \(AR \subseteq C \times A\) being the set of \textbf{aggregation relations} associating conceptual resources with instances of aggregation and grouping resources, hence instances of \texttt{skos:ConceptScheme} and \texttt{skos:Collection} or subclasses thereof.
\end{itemize}

\end{definition}


\subsection{Graph Metrics}

The commonality of the following metrics is that they apply a graph-based view on a given vocabulary \(v\), which we can then consider as a directed graph \(G=(N,E)\) consisting of a set of nodes \(N\) and edges \(E\).

\subsubsection{Loose Concepts} is a metric motivated by the notion of ``orphan terms'' in the literature~\cite{Hedden2010}, which are are terms without any associative or hierarchical relationships. Checking for them is common in existing thesaurus development applications and also suggested by \cite{Z39.19:2005}. When we apply this to SKOS vocabularies, a loose concept is a concept that has no semantic or aggregation relations with other conceptual, aggregation or grouping resources. Although they might have attached lexical labels, they lack valuable context information. The number of loose concepts in a vocabulary can be computed as follows:

% Note: this is generally true for all vocabularies, right? "Depending on the vocabulary development policy, loose concepts may be tolerated or considered to be a defect."

\begin{definition}
    
Let \(G = (C, SR \cup AR)\) and \(deg(c)\) be the degree, hence the number of in- and outgoing edges, of a node $c \in C$ in a graph $G$. We can then define loose concepts as a function \(lc : G \rightarrow \mathbb{N}_{0}\), with \[lc(g) = \left|\left\{c \in C : deg(c) = 0\right\}\right|\]

\end{definition}

% \subsubsection{Weakly Connected Components} cause the vocabulary to split into separate disconnected components. This is very similar to the ``loose concepts” criterion introduced before and might be caused by incomplete data acquisition, ``forgotten" test data, outdated terms, accidental deletion of relations and the like. In a practical setting, existence of weakly connected components could render the vocabulary less suitable for operations that rely on navigating a strongly connected vocabulary structure, such as query expansion or suggestion of related terms.
% 
% \begin{definition}
% To evaluate this criterion we consider a graph \(G = (AC, SRAR)\). The function \(components(g) : G \rightarrow  \mathfrak{P}(G)\) calculates all weakly connected components of a graph by replacing all directed edges by undirected edges and utilizing “Tarjan’s algorithm” \cite{Hopcroft1973}. Based on that function, we can express this criterion as a function \(wcc : G \rightarrow \mathbb{N}_{0}\), with \[wcc(g) = \left|components(g)\right|\]
% \end{definition}
% 
% \subsubsection{Cyclic Hierarchical Relations} have been mentioned numerous times in thesaurus development literature. \cite{Soergel2002} suggests a ``check for hierarchy cycles" since they ``throw the program for a loop in the generation of a complete hierarchical structure”. However, the SKOS documentation does not state, how exactly hierarchical relationships are to be interpreted. There exist common forms like, e.g., “generic-specific”, “instance-of” or “whole-part” \cite{Hedden2010,Harpring2010,Aitchison2000}. Hierarchical cycles for some of these interpretations in various domains or usage scenarios might be perfectly valid, whereas they would be treated as failures in others.
% 
% \begin{definition}
% In the context of this work, we define a cycle in a SKOS vocabulary as a logical contradiction in the broader/narrower hierarchy. Let \(HB \subseteq SR\) be the set of broader hierarchical relations associating two conceptual resources with instances of \texttt{skos:broader}, \texttt{skos:broaderTransitive}, \texttt{skos:broadMatch} or subproperties thereof. \(HB\) also associates conceptual resources related with instances \texttt{skos:narrower}, \texttt{skos:narrowTransitive}, \texttt{skos:narrowMatch} or subproperties thereof in reverse order. Likewise, \(HN \subseteq SR\) can be defined as the set of narrower hierarchical relations. We consider two graphs \(GB = (C,HB)\) and \(GN=(C,HN)\) and a function \(cycleNodes:GB \cup GN \rightarrow \mathfrak{P}(C)\) identifying all nodes in a graph, that are part of a cycle. Furthermore we consider a function \(stronglyConnectedSets: GB \cup GN \rightarrow \mathfrak{P}(C)\) that calculates all strongly connected undirected subgraphs of a graph. This criterion can be defined as a function \(chr:GB \cup GN \rightarrow \mathbb{N}_{0}\) with \[chr(g)=\left|\left\{s \in stronglyConnectedSets(g) : cn \in cycleNodes(g), cn \in s\right\}\right|\]
% \end{definition}

% \subsection{Structure-Centric Quality Criteria / Structural Measures}
% 
% \subsubsection{Redundant Associative Relations} \cite{ISO25964-1:2011} suggests that terms that share a common broader term but don’t have an overlapping meaning, should not be related associatively. This is also advocated by \cite{Hedden2010} and \cite{Aitchison2000} who also mentions ``the risk that thesaurus compilers may overload the thesaurus with valueless relationships", having a negative effect on precision.
% 
% \begin{definition}
% Let \(ASR \subseteq SR\) be the set of associative relations in SKOS, namely \texttt{skos:related} and \texttt{skos:relatedMatch}. These relations are symmetric, so \((c1,c2) \in ASR \Rightarrow (c2,c1) \in ASR\). Formally, this criterion can be expressed as a function \(rar:V \rightarrow \mathbb{N}_{0}\) with \[rar(v)=\left|\left\{(c1,c2) \in ASR : c3 \in C, (c1,c3) \in HB, (c2,c3) \in HB\right\}\right|\]
% \end{definition}
% 
% \subsubsection{Number of Hierarchically and Associatively Related Concepts}\label{harc} \cite{Aitchison2000} mentions a ``common error in manually produced thesauri" when concepts are related both hierarchically and associatively (F1.3.2). \cite{Hedden2010} also proposes to check for ``pairs of terms that have more than one kind of relationship" as a feasible rule, enforced by thesaurus management software.
% 
% % \begin{definition}
% % This criterion can be expressed as a function \(har:V \righarrow \mathbb{N}_{0}\) with \[har(v)=\left|\left\{c \in C : c' \in C, (c,c') \in HB \cup HN, (c,c') \in ASR\right\}\right|\]
% % \end{definition}
% 
% \subsubsection{Unidirectionally Related Concepts} \cite{Z39.19:2005} suggests thesaurus maintenance software to automatically add reciprocal relations (e.g., broader/narrower, related) to the resulting thesaurus. Vocabularies created in that manner are expected, e.g., to be better browsable for humans and provide better search results when used in systems without reasoning support.
% 
% \begin{definition}
% Let \(ISR \subseteq R \times R\) be a set that associates resources defined in \(V\) that are related by a SKOS property that has an inverse SKOS property defined. Furthermore, ISR also associates resources that are related by a symmetric SKOS property. In an ideal setting \((r,r') \in ISR \iff (r',r) \in ISR\). Thus we can express this criterion as function \(urc:V \rightarrow \mathbb{N}_{0}\) with \[urc(v)=\left|\left\{c \in C : r \in R, (c,r) \in ISR, (r,c) \notin ISR\right\}\right|\]
% \end{definition}
% 
% \subsubsection{Solely Transitively Related Concepts} Two concepts are related using only transitive hierarchical relations which are, according to the SKOS reference document, ``not used to make assertions". Transitive hierarchical relations in SKOS are meant to be infered by the vocabulary user, which is reflected in the SKOS schema by, e.g., broader being a subproperty of broaderTransitive.
% 
% \begin{definition}
% Let \(TR \subseteq SR\) be the set of all conceptual resources being transitively related by \texttt{skos:broaderTransitive} respectively \texttt{skos:narrowerTransitive}, not by subproperties thereof. Furthermore, let \(HR \subseteq SR\) be the set of all conceptual resources being related by \texttt{skos:broader} and \texttt{skos:narrower} with \((c,c') \in HR \Rightarrow (c',c) \in HR\). We can write this criterion as a function \(str:V \rightarrow \mathbb{N}_{0}\) with \[str(v)=\left|\left\{c \in C : c' \in C, (c,c') \in TR, (c,c') \notin HR\right\}\right|\]
% \end{definition}
% 
% \subsection{Linked Data Specific Criteria / Linked Data Measures}
% SKOS has been developed for creating knowledge organization systems that are used ``within the framework of the semantic web" \cite{Bernerslee2001}. Therefore we identified some criteria targeting issues related to best practises in publishing Linked Data, the underlying technology for establishing a Semantic Web \cite{Heath2011}.
% 
% \subsubsection{Concept External Link Average} The ``Web of Data'' consists of globally interconnected datasets, ``enabling seamless connections between data sets" \cite{Heath2011}. These links can be established between various SKOS vocabularies by ``external links", i.e., references from conceptual resources in a ``source vocabulary" to resources identified by an HTTP URI and available at a different host.
% 
% % \begin{definition}
% % Let \(EL=C \times R\) be the set of external links with \(host(c) \neq host(r)\). This criterion can then be written as a function \(elc:V \righarrow \mathbb{ℚ}\) with \[elc(v)=\frac{\left|EL\right|}{\left|C\right|}\]
% % \end{definition}
% 
% \subsubsection{HTTP URI Scheme Violation} One of the key concept of Linked Data is that URIs should be dereferencable in order to browse through data in the same way as through ``traditional" Web pages. Using HTTP URIs therefore is the logical consequence and mentioned numerous time in literature \cite{Heath2011}. However, some vocabularies use other URI schemes such as URNs and DOIs when defining resources, which impedes data lookup and crawling purposes.
% 
% \begin{definition}
% Let \(VR \subseteq C \cup A\) be the set of vocabulary resources identified by an URI (not by a blank node). We can define a function \(httpScheme:R \rightarrow \left\{0,1\right\}\) with \(httpScheme(r)=1\) if the scheme part of the URI is http or https, 0 otherwise. Then this criterion can be written as a function \(usv:V \rightarrow \mathbb{N}_{0}\) with \[usv(v)=\left|\left\{vr \in VR : httpScheme(vr)=0\right\}\right|\]
% \end{definition}
% 
% \subsubsection{Link Target Unavailability} Just as in the ``traditional" Web, resources in the Web of Data might also be unavailable, i.e. resolving their URI leads to an erroneous HTTP response or no response at all. An erroneous HTTP response in that case can be defined as a response code (after a possible redirection) other than 200. Since these ``broken links" hinder information gathering, they should be avoided.
% 
% \begin{definition}
% Let \(LR \subseteq R\) be the set of all resources being related to conceptual, aggregational and grouping resources. The function \(resp:R \rightarrow \mathbb{N}\) assigns a resource its HTTP response code after following all redirections. We can therefore write this criterion as function \(ltu:V \rightarrow \mathbb{N}_{0}\) with \[ltu(v)=\left|\left\{r \in LR \cup VR : resp(r)=200\right\}\right|\]
% \end{definition}
% 
% \subsubsection{Average Concept In-degree} A periodically updated version of a diagram visualizing ``the data sets in the LOD cloud as well as their interlinkage relationships” is published online\footnote{http://www4.wiwiss.fu-berlin.de/lodcloud/state/}. In these diagrams, DBpedia \footnote{http://dbpedia.org/About} is shown as the most intensely linked resource, i.e. a large number of vocabularies link to resources in that dataset. Therefore a dataset with a large number of other datasets referencing it, can be assumed to be of great value for the community. Accordingly, estimating the number of ``incoming links" of SKOS concepts defined in a vocabulary (henceforth called the in-degree of a concept) would give some hint about how established or accepted a controlled vocabulary is. A method to carry out these estimations would be utilizing existing Linked Data indices like, e.g., Sindice\footnote{http://sindice.com/} that provide a SPARQL endpoint.
% 
% \begin{definition}
% Let \(U\) be the universe of all datasets on the Web and \(V \in U\). The set \(RR \subseteq R \times R\) contains all resources in distinct datasets related by an RDF property with \((r1,r2) \in RR \Rightarrow host(r1) \neq host(r2)\). This criterion can be written as a function \(aci:V \rightarrow \mathbb{ℚ}\) with \[aci(v)=\frac{\left|\left\{r \in R : c \in C, (r,c) \in RR\right\}\right|}{\left|C\right|}\] 
% \end{definition}
% 
% \subsection{SKOS-Specific Criteria / SKOS Semantic Measures}
% \subsubsection{Associative vs. Hierarchical Relation Clashes} This criterion is similar to criterion ??\ref{harc}??, however it strictly covers integrity condition S27 from the SKOS reference document, i.e. involving the transitive hull of hierarchical relations, not taking into account hierarchical mapping properties.
% 
% % \begin{definition}
% % Let \(CBR \subseteq SR\) and \(CNR \subseteq SR\) be the set of conceptual resources being hierarchically broader respectively narrower related within the same concept scheme by the properties \texttt{skos:broader} and \texttt{skos:narrower}. Furthermore, the set \(REL \subseteq SR\) contains all conceptual resources related by the \texttt{skos:related} property. We consider two directed graphs \(GB=(C,CBR)\) and \(GN=(C,CNR)\) and a function hierarchical:C \times C \rightarrow \left\{0,1\right\} with \(hierarchical((c,c'))=1\) if a path between c and c’ exists in either \(GB\) or \(GN\), 0 otherwise. This criterion can then be written as a function \(ahrc:G \rightarrow \mathbb{N}_{0}\) with \[ahrc(g)=\left|\left\{(c,c') \in REL:hierarchical((c,c'))=1\right\}\right|\]
% % \end{definition}
% 
% \subsubsection{Exact vs. Associative and Hierarchical Mapping Clashes} Integrity condition S46 of the SKOS reference document states disjointness of the property \texttt{skos:exactMatch} with \texttt{skos:broadMatch} and \texttt{skos:relatedMatch}.
% 
% % \begin{definition}
% % Let \(EM \subseteq SR\), \(BM \subseteq SR\), \(RM \subseteq SR\) be the set of conceptual resources related by \texttt{skos:exactMatch}, \texttt{skos:broadMatch}, and \texttt{skos:relatedMatch}. In functional form, this criteria can be written as \(eahmc:V \rightarrow \mathbb{N}_{0}\) with \[eahmc(v)=\left|\left\{em \in EM : em \in BM \cup RM\right\}\right|\]
% % \end{definition}
% 
% \subsubsection{Illegal SKOS Terms} Vocabularies expressed in the SKOS language can be easily be extended by proprietary classes or properties in order to meet certain requirements of the publishing organizations that cannot (yet) be expressed within the SKOS schema. However, there seems to exist a misunderstanding in some vocabularies about the correct way of introducing new classes or properties in an OWL document. In some cases, vocabularies ``invent" new terms in the SKOS namespace that cannot be resolved. This criterion has been identified as ``Use of undefined classes and properties" in \cite{Hogan2010}.
% 
% \subsubsection{Deprecated Property Usage} With the SKOS W3C recommendation maturing, some properties have been removed from the current version of the schema. Vocabularies using these ``deprecated" properties hence contain parts of their information in a non ``standard"-compliant way, having negative effects when, e.g., crawling vocabularies to gather specific information.
% 
% % \begin{definition}
% % We define a set that contains all deprecated properties DP=\left\{\texttt{skos:symbol}, \texttt{skos:prefSymbol}, \texttt{skos:altSymbol}, \texttt{skos:CollectableProperty}, \texttt{skos:subject}, \texttt{skos:isSubjectOf}, \texttt{skos:primarySubject}, \texttt{skos:isPrimarySubjectOf}, \texttt{skos:subjectIndicator}\right\}. Let \(DR \subseteq RV \times RV\) with \(RV \subseteq R\) being the set of all RDF resources in V that are related by a property in DP. This criterion can then be written as a function \(dpu:V \rightarrow \mathbb{N}_{0}\) with \[dpu(v)=\left|DR\right|\]
% % \end{definition}
% 
% \subsubsection{Omitted Top Concepts} The SKOS language provides ConceptSchemes which are a facility for grouping related concepts. In order to provide entry points to such a group of concepts, one or more concepts can be marked as top concepts. This helps to provide “efficient access”\footnote{SKOS primer, \url{http://www.w3.org/TR/skos-primer/}} and simplifies orientation in the vocabulary.
% 
% \begin{definition}
% Let \(TCR \subseteq AR\) be the set of concepts related to a \texttt{skos:ConceptScheme} by the properties \texttt{skos:topConceptOf} or \texttt{skos:hasTopConcept}. Furthermore, \(CS \subseteq A\) is the set of ConceptSchemes defined in a vocabulary. We can then define this criterion as a function \(otc:V \rightarrow \left\{0,1\right\}\) with \[otc(v)=0 \iff \forall cs \in CS. \exists (c,cs) \in TCR, 1 otherwise\]
% \end{definition}
% 
% \subsubsection{Top Concepts Having Broader Concepts} \cite{Allemang2011} proposes to “not indicate any concepts internal to the tree as top concepts”, i.e. top concepts in a concept scheme should not have broader concepts.
% 
% \begin{definition}
% Let \(CBR \subseteq SR\) be the set of conceptual resources having broader concepts within the same concept scheme, i.e. being related to another concept by the property \texttt{skos:broader}. \(tcb:V \rightarrow \mathbb{N}_{0}\) with \[tcb(v)=\left|\left\{c \in C : (c,cs) \in TCR, (c,c') \in CBR\right\}\right|\]
% \end{definition}
% 
% \subsection{Labeling Issues}
% 
% \subsection{Other Criteria}


%!TEX root = main.tex

\section{Analysis of Existing SKOS Vocabularies}\label{sec:analysis}

To learn about possible quality issues in real-world vocabularies, we implemented the previously described quality checking function in the qSKOS tool and applied each function on a set of existing SKOS vocabularies. 

\subsection{The qSKOS Quality Checking Tool}

The open-source qSKOS\footnote{\url{https://github.com/cmader/qSKOS/}} quality assessment tool can be used to find possible quality issues a given SKOS vocabulary. Users can run the tool by passing a vocabulary and selecting the quality checking functions they want to perform. As a result, they obtain detailed reports listing possibly affected resources alongside with the potential cause of the issue. qSKOS is implemented in Java and can be used as standalone command-line tool or API in any other application. Its design is open for the introduction of additional quality functions. In our analysis, we used qSKOS version 0.2., which can be downloaded from: \url{https://github.com/cmader/qSKOS/releases/qSKOS-0.2.tar.gz}.

\todo{BH}{I think we can remove the collections column, none of the issues involve collections}

\subsection{Vocabulary Data Set}

We used Sindice and available listings\footnote{\url{http://www.w3.org/2001/sw/wiki/SKOS/Datasets}} to learn about existing SKOS vocabularies. Table~\ref{tab:vocabs} lists our representative vocabulary selection and shows that the vocabularies differ in size and application domain: the \emph{Gemeenschappelijke Thesaurus Audiovisuele Archieven (GTAA)} is vocabulary from the media domain, \emph{Eurovoc} and the \emph{Integrated Public Sector Vocabularies (PSV)} cover public sector concepts, the \emph{Medical Subject Headings (MeSH)} and \emph{Peroxisome Knowledge Base (PXV)} are from the (bio-)medical domain, the \emph{Geonames Ontology} covers locations, the \emph{Thesaurus for Economics (STW)} and \emph{North American Industry Classification System (NAICS)} are vocabularies from the economics and business area and the \emph{DBPedia Categories} reflect Wikipedia's user-generated categorization system. We also include the \emph{Meketre} vocabulary, which defines Egyptology-related concepts, and the non-public \emph{Austrian Armed Forces Thesaurus (LVAk)}, which we generated from CSV files provided by the vocabulary maintainers. A zip-file containing all 14 public vocabulariesx we used in our analysis, can be downloaded from: \url{http://www.github.com/cmader/???}. In contrast to the column ``Concepts'' that shows the total number of SKOS Concepts that can be infered from the vocabulary, the column ``Authoritative Concepts`` lists only those concepts that have an URI in the namespace of the vocabulary. These concepts are of importance when finding Out-Links to other vocabularies on the LOD cloud.

\begin{table}
\label{tab:vocabs}
\caption{Analyzed SKOS vocabularies}
    
\begin{center}
\resizebox{\textwidth}{!} {
\setlength{\extrarowheight}{5pt}

\begin{tabular}{p{6cm}ccccccccc}

\textbf{Vocabulary} & \rotatebox{90}{\textbf{Abbreviation}} & \rotatebox{90}{\textbf{Version/last mod.}} & \rotatebox{90}{\textbf{Concepts}} & \rotatebox{90}{\textbf{Auth. Concepts}} & \rotatebox{90}{\textbf{Labels}} & \rotatebox{90}{\textbf{Semantic Rel.}} & \rotatebox{90}{\textbf{Aggregation Rel.}} & \rotatebox{90}{\textbf{Concept Schemes}} & \rotatebox{90}{\textbf{Collections}}\\
\toprule
Agricultural Thesaurus & \textbf{AGROVOC} & 1.3 & 32,035 & 32,035 & 620,629 & 65,934 & 32,085 & 1 & 0 \\
\hline
DBpedia Categories & \textbf{DBpedia} & ? & 743,410 & 743,410 & 740,352 & 1,490,316 & 0 & 0 & 0 \\
\hline
The EU's multilingual thesaurus & \textbf{Eurovoc} & 5.0 & 6,797 & 6,797 & 457,788 & 18,491 & 15,512 & 128 & 0 \\
\hline
Geonames Ontology & \textbf{Geonames} & 2.2.1 & 671 & 671 & 671 & 0 & 671 & 9 & 0 \\
\hline
Gemeenschappelijke Thesaurus Audiovisuele Archieven & \textbf{GTAA} & 2010/08/25 & 171,991 & 171,991 & 178,776 & 50,892 & 343,980 & 9 & 0 \\
\hline
Integrated Public Sector Vocabulary & \textbf{IPSV} & 2.00 & 4,732 && 7,945 & 13,843 & 4,483 & 3 & 0 \\
\hline
Library of Congress Subject Headings & \textbf{LCSH} & 2011/08/09 & 459,182 & 407,908 & 746,076 & 595,754 & 815,816 & 19 & 0 \\
\hline
Austrian Armed Forces Thesaurus & \textbf{LVAk} & n/a & 13,411 & 13,411 & 17,250 & 16,346 & 0 & 0 & 0 \\
\hline
Middle Kingdom tombs of Ancient Egypt Thesaurus & \textbf{Meketre} & 2011/07/07 & 422 & 422 & 569 & 1,698 & 6 & 2 & 0 \\
\hline
Medical Subject Headings & \textbf{MeSH} & ? & 24,626 & 24,626 & 150,617 & 38,858 & 0 & 0 & 0 \\
\hline
North American Industry Classification System & \textbf{NAICS} & 2012 & 4,175 & 2,213 & 0 & 8,684 & 2,235 & 1 & 0 \\
\hline
New York Times People & \textbf{NYTP} & 2010/06/22 & 4,979 & 4,979 & 4,979 & 0 & 4,979 & 1 & 0 \\
\hline
University of Southampton Pressinfo & \textbf{Pressinfo} & 2011/02/24 & 1,125 & 1,125 & 0 & 0 & 0 & 0 & 0 \\
\hline
Peroxisome Knowledge Base & \textbf{PXV} & 1.6 & 2,112 & 1,686 & 3,628 & 2,695 & 1,716 & 1 & 0 \\
\hline
Thesaurus for Economics & \textbf{STW} & 8.06 & 6,524 & 6,524 & 31,189 & 57,907 & 6,531 & 1 & 0 \\
\bottomrule
\end{tabular}

}
\end{center}
\end{table}

\subsection{Results}

The results of this analysis are reported in Table~\ref{tab:results}, which shows the absolute number of possibly affected resources for each quality checking function and vocabulary combination. The value 0 means that we haven't found any affected resource, N/A means that a certain function was not applicable for reasons we explain below. An asterisk after certain numeric values indicates that due to performance reason, only an extrapolated subset containing 5\% of the respective elements (HTTP URIs or concepts) of the vocabulary has been analyzed.

\begin{table}[h]
\label{tab:results}
\caption{Results of the quality checking functions}

\begin{center}
\resizebox{\textwidth}{!} {
\setlength{\extrarowheight}{5pt}

\begin{tabular}{p{4cm}ccccccccccccccc}
\textbf{Issue} & \rotatebox{90}{\textbf{GTAA}} & \rotatebox{90}{\textbf{Geonames}} & \rotatebox{90}{\textbf{MeSH}} & \rotatebox{90}{\textbf{PXV}} & \rotatebox{90}{\textbf{Eurovoc}} & \rotatebox{90}{\textbf{IPSV}} & \rotatebox{90}{\textbf{Agrovoc}} & \rotatebox{90}{\textbf{DBpedia}} & \rotatebox{90}{\textbf{Pressinfo}} & \rotatebox{90}{\textbf{NYTP}} & \rotatebox{90}{\textbf{LCSH}} & \rotatebox{90}{\textbf{Meketre}} & \rotatebox{90}{\textbf{STW}} & \rotatebox{90}{\textbf{NAICS}} & \rotatebox{90}{\textbf{LVAk}} \\
\toprule
Omitted or Invalid Language Tags & 0 & 0 & 23,950 & 1,578 & n/a & 0 & 0 & 0 & 1,224 & 0 & 18 & 0 & 2 & n/a & 13,411 \\

Incomplete Language Coverage & 0 & 0 & 0 & 0 & n/a & 0 & 32,035 & 0 & 0 & 0 & 0 & 420 & 6,456 & n/a & 0 \\

Undocumented Concepts & 96,850 & 0 & 1,807 & 1,918 & 5,341 & 4,551 & 32,035 & 743,410 & 1,125 & 4,094 & 398,036 & 422 & 5,236 & 3,259 & 13,411 \\

Label Conflicts & 12,404 & 18 & 0 & 7 & n/a & 0 & 2,949 & 0 & 0 & 0 & tba & 4 & 5 & n/a & 13 \\

\midrule

Orphan Concepts & 162,000 & 671 & 0 & 2 & 7 & 0 & 0 & 77,062 & 1,125 & 4,979 & 172,364 & 0 & 4 & 0 & 21 \\

Weakly Connected Components & 621 & 0 & 4 & 10 & 4 & 1 & 4 & 1,506 & 0 & 0 & 22,131 & 5 & 1 & 1 & 11 \\

Cyclic Hierarchical Relations & 0 & 0 & 4 & 0 & 0 & 0 & 0 & 1,132 & 0 & 0 & 0 & 0 & 0 & 0 & 5 \\

Valueless Associative Relations & 9,438 & 0 & 495 & 0 & 1 && 282 & 8,120 & 0 & 0 & 1,879 & 0 & 5,082 & 0 & 5 \\

Solely Transitively Related Concepts & 0 & 0 & 0 & 0 & 2,652 && 0 & 0 & 0 & 0 & 0 & 36 & 0 & 2,189 & 0 \\

Omitted Top Concepts & 9 & 9 & 0 & 0 & 1 && 0 & 0 & 0 & 1 & 18 & 0 & 0 & 0 & 0 \\

Top Concept Having Broader Concepts & 0 & 0 & 0 & 1 & 0 && 0 & 0 & 0 & 0 & 0 & 0 & 0 & 0 & 0 \\

\midrule

Missing In-Links && 19 && 1,686 &&&&& 1,125 & 20 && 422 & 6,516 & 2,213 & 13,411 \\

Missing Out-Links & 171,991 & 671 && 1,472 & 6,797 && 32,035 & 743,410 & 1,116 & 0 & 408,198 & 273 & 6,524 & 0 & 13,411 \\

Broken Links & 0 & 0 & 1 & 163 &&& 238 & 0* & 11 & 7 && 425 & 1 & 3,169 & n/a \\


Undefined SKOS Resources & 0 & 0 & 1 & 0 & 0 & 1 & 0 & 0 & 0 & 0 & 0 & 0 & 0 & 0 & 0  \\

\bottomrule
\end{tabular}
}
\end{center}
\end{table}

% Labeling and Documentation Issues
We found labeling and documentation issues in all analyzed vocabularies. PXV provides labels for XX\% of its concepts, Pressinfo labels only for 1 out of 1125 concepts, and the STW thesaurus misses labels for two concepts. Eurovoc and NAICS use SKOS-XL labels, which are currently not supported by qSKOS.
Incomplete Language Coverage were issues in the three analyzed vocabularies: in AGROVOC all concepts had incomplete coverage, probably because \todo{CM}{es gibt labels in insgesamt 25 versch sprachen und keines der konzepte ist in jeder sprache beschrieben}, and in STW we found XX\%.
Most vocabularies didn't document their concepts with the documentary properties defined by SKOS.
Label conflicts, hence \todo{CM}{explain briefly} were found in several vocabularies: GTAA has many because? PXV has 7. Why given examples?

% Structural Issues


% Linked Data Specific Issues


% The \textbf{Loose Concepts} metrics provides insights on the structural complexity of vocabularies. The large number of loose concepts in GTAA can be explained by the fact that it contains names (e.g., persons or places) that arent't connected using skos properties. This is also the case for Geonames, NYTP and Pressinfo where the number of total conceps is equal to the number of loose concepts, indicating that no concept is semantically related to another concept.
% 
% The large number of \textbf{Weakly Connected Components} in GTAA is caused by many ``minimal'' components, containing only 2 concepts connected by a \texttt{skos:related} property. Due to the fact that loose concepts are not counted as weakly connected components, Geonames, NYTP and Pressinfo evaluate to zero components.


%!TEX root = main.tex
\section{Related Work}\label{sec:related_work}

The problem of ``vocabulary quality'' is closely related to the more general problem of “data quality”, which has intensively been discussed in data and information systems research. \cite{Pipino2002,Batini2009} argue that dealing with data quality should involve both “subjective perceptions of the individuals” and “objective measurements based on the data set in question”. We see our work as a contribution to the latter and believe that the results of SKOS quality measures must be combined with domain knowledge, and therefore human expertise, to make final quality judgements.

A vast number of publications in the field of designing, validating and testing controlled vocabularies is available. Many of them propose properties \cite{Soergel1995}, design guidelines \cite{Svenonius2003,...} or metrics \cite{Elkin2002,Kless2010} aiming to improve, e.g., retrieval precision and recall, consistency and multilinguality. The work is based on various sources like, e.g., review of existing vocabularies~\cite{Soergel1995} or survey-based studies~\cite{Pinto2008}. However, there are hardly any formally defined quality indicators that can be automatically assessed without further knowledge about the application domain, targeted user group or usage scenario.

%For controlled vocabularies, \cite{Svenonius2003} argues precision and recall to be “the chief objectives of any retrieval language” and proposes a set of design guidelines (e.g., term selection, structural and syntactical considerations) aiming to improve these objectives. Elkin and XXX? [Elkin2002] identify general quality metrics and structural and maintainability considerations for controlled vocabularies in the health domain, which cover issues like non-redundancy, consistency, and multilinguality of concepts? Soergel [Soergel1995] mentions properties of “good” thesauri like, e.g., support for synonym and hierarchic expansion or request-oriented indexing. As negative properties of the AAT Soergel points out, e.g., missing cross-references, incomplete facet analysis, the monohierarchical layout and shortcomings in term form and choice. On the positive side he mentions, e.g., presence of scope notes, definition of related terms and some definitions of synonyms. Pinto [Pinto2008] performs a survey-based study comparing perceived and expected properties (“variables”) of thesauri used in social science databases. The findings exposed need for “considerable improvement” in the structural perspective (e.g., equivalence and associative relationships), but also regarding “performance” variables (e.g., explanatory notes or expected performance) and “format” issues (e.g., ergonomics and display). Some of the identified variables (e.g, pre-coordination, searching options) are hard do assess automatically or are application dependent, others aren’t exactly defined or formalized. However, neither of them suggest any automatic assessment methods, which can be applied to controlled Web vocabularies.

<<<<<<< HEAD
Research on the notion of “data quality” has also been conducted in Semantic Web research. For general datasets, \cite{Heath2011,Hogan2010} propose best practices and ..   \cite{Tartir2007} 

Hogan and ??? [Hogan2009] identify four categories of common errors and shortcomings in RDF documents. Some of them are also of great importance and can be adopted for assessing the quality of SKOS vocabularies, especially considering interoperability issues like e.g., dereferencability and use of undefined classes and properties. However, the author’s focus lies on RDF datasets in general, thus features of controlled vocabularies are not covered. This is also the case for [Tartir2007] who propose a framework for evaluation and ranking of Ontologies. Given the fact that vocabularies usually extend the SKOS schema only to a very limited degree by defining subclasses and subproperties, the metrics defined in [Tartir2007] are not considered applicable to these vocabularies.
=======
Research on the notion of “data quality” has also been conducted in Semantic Web research. Hogan and ??? [Hogan2009] identify four categories of common errors and shortcomings in RDF documents. Some of them are also of great importance and can be adopted for assessing the quality of SKOS vocabularies, especially considering interoperability issues like e.g., dereferencability and use of undefined classes and properties. However, the author’s focus lies on RDF datasets in general, thus features of controlled vocabularies are not covered. This is also the case for [Tartir2007] who propose a framework for evaluation and ranking of Ontologies. Given the fact that vocabularies usually extend the SKOS schema only to a very limited degree by defining subclasses and subproperties, the metrics defined in [Tartir2007] are not considered applicable to these vocabularies.
>>>>>>> 2a5aecf06681ee581815e0968fdf0401784682ee

[Arpinar2006] define three types of conflicts that may occur in ontologies. Based on these types, the ontology maintainers can create rules in RuleML or SWRL to find violations in their ontologies. Although not using a rule language, we express quality criteria formally, following a very similar approach.

[Gangemi2005] introduce three “measure types for ontology evaluation” (dimensions), i.e. structural, functional and usability-related measures. The structural exploits the graph structure of an ontology and defines a number of measure like, e.g., absolute, average or maximum depth and breadth, number of leaf nodes, number of siblings or density. While for many of the measures, its applicability for controlled vocabularies is not immediately clear, some (e.g., cycle ratio, inverse relations) resemble common patterns in controlled vocabularies and thus are represented in our catalog of quality criteria. Functional dimensions are measures related to how well an ontology fits its intended purpose like, e.g., recall, precision and accuracy. They are highly context-dependent, however, we expect most of our identified criteria to impact these dimensons. In the usability dimension, Gangemi et al. mention three analytical levels for usability profiling: recognition, efficiency, and interfacing. Since they don’t go into details on what the implications for thesauri are, we try to complement this in the rationale of our identified quality crtieria. [Brank2005] provide a concise overview on ontology evaluation approaches
Linked Data Quality

[Yeganeh2011] propose a method to convert semi-structured data (e.g., XML files) into “high-quality Linked Data”. Their notion of quality encompasses usage of HTTP URIs, URI dereferencability, linkage of related objects as well as merging and detection of duplicate resources.


%!TEX root = main.tex
\section{Conclusions and Future Work}\label{sec:conclusions}

% make clear that this shouldn't replace taxonomists

% web app

% already improvements in two vocabularies

\todo{ALL}{discuss implications of OWA on SKOS quality.}


% We skip this for the submission:

% \section{Acknowledgements}
% Tom Dent (esd support team) for providing a dump for the IPSV vocabulary
% Andrew Gibson for providing a dump for the Peroxisome knowledge Base and giving valuable feedback on their vocabulary structure.


\bibliography{references}
\bibliographystyle{splncs03}

\end{document}

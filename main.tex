\documentclass{llncs}

\usepackage{url}

\usepackage{verbatim}
\usepackage{amssymb}
\usepackage{rotating}
\usepackage{booktabs}

\usepackage{color}

\newcommand{\todo}[2]{\textbf{\textcolor{red}{(TODO [#1]: #2)}}}

% metrics = quantitative indicator of sth

\title{Quality Criteria for SKOS Vocabularies}
\author{Christian Mader\inst{1} \and Bernhard Haslhofer\inst{2}}
\institute{
	University of Vienna, Faculty of Computer Science\\\email{christian.mader@univie.ac.at}
	\and Cornell University, Information Science\\\email{bernhard.haslhofer@cornell.edu}}

\begin{document}

\maketitle

\begin{abstract}
    
The Simple Knowledge Organization System (SKOS) has become a standard model for controlled vocabularies on the Web. However, SKOS vocabularies often differ in terms of quality, which reduces their applicability across system boundaries. Here we investigate how we can support taxonomists in improving the quality of SKOS vocabularies by pointing out quality issues that go beyond the integrity constraints defined in the SKOS specification. We identified possible \textbf{quality issues} and formalized them into computable \textbf{quality checking functions} that can find affected resources in a given SKOS vocabulary. We implemented these functions in the qSKOS quality assessment tool, analyzed 15 existing vocabularies, and found possible quality issues in all of them.

\end{abstract}

%!TEX root = main.tex

\section{Introduction}\label{sec:introduction}

% BACKGROUND

% What is SKOS?
The Simple Knowledge Organization System (SKOS)~\cite{SkosReference2008} is a standard model for sharing and linking controlled vocabularies (thesauri, classification systems, etc.) on the Web. Many organizations, including the European Union\footnote{EuroVoc, \url{http://eurovoc.europa.eu/}}, the United Nations\footnote{AGROVOC, \url{http://aims.fao.org/website/AGROVOC-Thesaurus/sub}}, or the UK government\footnote{Integrated Public Sector Vocabulary (IPSV), \url{http://doc.esd.org.uk/IPSV}} publish SKOS representations of their vocabularies on the Web so that they can easily be accessed by humans and machines.

However, quality issues can affect the applicability of SKOS vocabularies for tasks such as query expansion, faceted browsing, or auto-completion, as in the following examples: 

\begin{itemize}

	\item AGROVOC defines concepts in 25 different languages. However, while most concepts have English labels attached, only 38\% have German labels. This can be a problem for multilingual applications that rely on label translations.

	\item An earlier version of the STW thesaurus contained 5 pairs of concepts with identical labels. As a result, the auto-complete function of the online search interface suggested identical entries without disambiguation information.

	\item The non-public vocabulary of the Austrian Armed Forces (LVAk) contains 11 disconnected concept clusters. When confronted with these structures, the thesaurus maintainers recognized them as ``forgotten'' test data that has no practical significance.
	
\end{itemize}

% PROBLEM / RESEARCH QUESTION

The SKOS specification defines a set of \textbf{integrity conditions} that state whether given data patterns are consistent with the SKOS model. Yet the SKOS integrity conditions fail to capture quality aspects like the ones above. The main reason lies in SKOS' ``minimal commitment'' approach. A standard that aims at cross-domain interoperability should refrain from defining constraints that impose on one domain the requirements of another. SKOS is thus very liberal with respect to data integrity.

On the other hand, each vocabulary should fulfill domain- and application-specific quality aspects and taxonomists often follow standard guidelines specific to given types of vocabularies (cf.,~\cite{ISO25964-1:2011,Z39.19:2005}) for thesauri) or apply their own hand-crafted checks~\cite{Coronado2009}. Existing guidelines consider these aspects, but currently rely on \textbf{human judgement}, which is subjective and does not scale for larger vocabularies. The SKOS context, where vocabularies can be linked together on the Web, also brings issues hitherto unforeseen by traditional checking approaches.

% CONTRIBUTION

We aim at contributing to the ongoing community efforts to bridge that gap between model-level integrity constraints and domain-specific quality aspects. Our goal is to help taxonomists in identifying possible quality issues in SKOS vocabularies and to give them a set of computable quality checking functions that, in combination with the taxonomists' experience and domain expertise, can serve as quality indicators for vocabularies. Finding such quality issues also gives important feedback on the overall vocabulary design process and should, at the end, lead to better vocabularies. The main contributions of this work can be summarized as follows:

\begin{itemize}

	\item We identified 15 \textbf{quality issues} for SKOS vocabularies by examining existing guidelines and formalized them into computable \textbf{quality checking functions} that identify possibly affected resources in a vocabulary.
	
	\item With the \textbf{qSKOS quality assessment tool} we provide a reference implementation of these functions.

	\item We \textbf{tested} these functions by \textbf{analyzing} a representative set of \textbf{15 existing SKOS vocabularies} to learn about possible quality issues.

\end{itemize}

In the following, we will first discuss what \emph{quality} means in the context of SKOS vocabularies and how it is currently support by the SKOS specification and existing tools. Then we introduce the quality issues we have identified and describe how we implemented them in the qSKOS quality assessment tool. Finally, we report on the results of an analysis we performed on 15 existing SKOS vocabularies and which shows that the quality issues we discussed are real and can lead to the improvement of existing vocabularies.

%!TEX root = main.tex

\section{Background and Methodology}

% About Quality in SKOS vocabularies

SKOS is a language for defining vocabularies in the Web of Data and therefore based on the Open World Assumption. Established quality notions from closed-world systems, such as referential integrity or schema validation, don’t hold anymore, because the available information may be incomplete and facts that are not explicitly stated cannot be determined as true or false. While trust and provenance models for Web data are being developed~\cite{Omitola2011,Hartig2009}, content-based and hand-crafted heuristics are currently used to evaluate quality in Linked Data sets~\cite{Heath2011}.

The quality of a given SKOS vocabulary can have a direct impact on the following application areas of controlled vocabularies: \todo{CM}{write a paragraph}

SKOS consistency and integrity means that
- 


\todo{CM}{After the introduction it is very common to introduce the problem in more detail. We should write a page about quality in controlled vocabularies and explain the sources for our quality metics. Maybe you can pull parts from the related work section.}

% Methodology

\todo{CM}{How did we identify the criteria we present in the following section?}



%!TEX root = main.tex

\section{Quality Criteria for SKOS Vocabularies}\label{sec:criteria}

% Briefly describe methodology

We identified an initial set of quality criteria for SKOS vocabularies by reviewing existing literature and manually examining existing vocabularies. We published our findings online\footnote{\url{https://github.com/cmader/qSKOS/wiki/Quality-Criteria-for-SKOS-Vocabularies}} and requested feedback from experts via public mailing list and informal face to face discussions. Based on the received responses, we then selected a subset of these criteria and formalized them into computable \textbf{quality checking functions}. Each function operates on a SKOS vocabulary, which we define, for the purpose of this work, as follows:

\begin{definition}[SKOS Vocabulary] Let a SKOS vocabulary be a tuple of the form $V = \langle IR, C, SR, LV \rangle$, with \(IR = I_{CEXT}(\texttt{rdfs:Resource}^\mathcal{I})\) being the set of \textbf{resources}, \(C = I_{CEXT}(\texttt{skos:Concept}^\mathcal{I})\) being the set of \textbf{conceptual resources}, $LV = I_{CEXT}(\texttt{rdfs:Literal}^\mathcal{I})$ being the set of \textbf{literal values}, and \item \(SR = I_{EXT}(\texttt{skos:semanticRelation}^\mathcal{I})\) being the set of \textbf{semantic relations} associating conceptual resources with one another. Further, we let $V$ be the fully entailed RDFS interpretation~\cite{RDFSEM2012} of the underlying RDF graph, enriched by entailment of \texttt{owl:inverseOf} properties. 

\end{definition}

The commonality of all quality checking functions is that they take a given SKOS vocabulary as input and return a (nested) set, containing the resources fulfill the quality criterion. Therefore, we can formalize a quality criterion as being an abstract quality checking function $f:V \rightarrow \mathcal{P}(IR)$, mapping a given vocabulary $v \in V$ to the powerset of its resources. Each quality criterion we found can be implemented as a realization of such an abstract filtering function and statistical analysis can be computed on top of the return values.

% Structure for each criterion description
%- origin (refs) and design rationale
%- formal definition that specifies how to compute the metric

In the following, we explain the origins and design rationale for each quality criterion and describe how we implemented the corresponding quality checking function. For better readability we provide semi-formal definitions only where necessary and refer to the online documentation or the implementation source code for further details. We also categorized the identified quality criteria into \emph{Labeling and Documentation Criteria}, \emph{Structural Criteria}, \emph{Linked Data Specific Criteria}, and \emph{Other Criteria}. 

% LEXICAL AND DOCUMENTATION CRITERIA

\subsection{Labeling and Documentation Criteria}

The criteria defined in this section primarily focus on supporting interaction with human users that either work with the vocabulary itself or interact with a system using the vocabulary for, e.g., search and retrieval purposes.

The criteria defined in this section focus 

\subsubsection{Omitted Language Tags}
Some controlled vocabularies contain literals in natural language without information about what language has actually been used. Using private language tags (see RFC3066) or omitting them at all limits, e.g., language-dependent queries. To identfy omitted language tags we iterate over all conceptual resources in $C$ and collect those that have relations to untyped literals lacking language information as described above. 

\subsubsection{Incomplete Language Coverage}
According to the previous criterion, vocabularies should combine plain literals with language tags. Furthermore, to support, e.g., internationalization or translation use cases, every concept should have defined literal values in the same set of languages than the other concepts. Calculation of the affected resources is done by iterating over all conceptual resources in $C$, building a global list of distinct languages throughout all literals assigned to every concept. Additionally, a list of used languages per conceptual resource is created. The criterion identifies all conceptual resources having a language list not equal to the global language list. 

\subsubsection{Undocumented Concepts}
\cite{Aitchison2000} cites Svenonius to advocate ``inclusion of as much definition material as possible''. The SKOS language specification defines various properties that are intended to hold this kind of information, subsumed as ``documentation properties'' in the SKOS reference. To identify the affected conceptual resources, we iterate over all elements in $C$ to find those that are not related to one of the documentation properties.

\subsubsection{Potentially Semantically Related Concepts} \todo{All}{reformulate criterion title}
This criterion is an extension of a statement from the SKOS primer, recommending ``no two concepts have the same preferred lexical label in a given language when they belong to the same concept scheme''. Given the fact that (i) many vocabularies don’t make use of concept schemes and (ii) also identical altLabels or hiddenLabels can have a negative effect in some use-cases (e.g, auto-completion of concept labels based on user input), we follow a more general approach: identification of all pairs of concepts with their respective prefLabel, altLabel or hiddenLabel meeting a threshold of a certain similarity function $sim:LV \times LV \rightarrow [0,1]$. For the analysis of the criterion in this document, we define the similarity function  to check for case-insensitive string equality and the threshold to equal 1. Calculation of this criterion is done by applying the similarity function to every label of every possible pair of conceptual concepts in $C$ and filtering those that meet or exceed a specified threshold.

% STRUCTURAL METRICS

\subsection{Structural Criteria}

The commonality of the following metrics is that they identify possible weaknesses when $V$ is used in scenarios exploiting the structure of its RDF graph. 

\subsubsection{Loose Concepts} is a criterion motivated by the notion of ``orphan terms'' in the literature~\cite{Hedden2010}, which are are terms without any associative or hierarchical relationships. Checking for them is common in existing thesaurus development applications and also suggested by \cite{Z39.19:2005}. When we apply this to SKOS vocabularies, a loose concept is a concept that has no semantic relations with other conceptual resources and thus do not occur in any of the pairs yielded by $SR$. Although they might have attached lexical labels, they lack valuable context information that is considered essential for, e.g., search query expansion. Calculation can be done by iterating over all elements in $C$, filtering those not having a semantic relation to any other conceptual resource. 

% \begin{definition}
% Let \(G_{lc} = (C, SR)\) and \(deg(c)\) be the degree, hence the number of in- and outgoing edges, of a node $c \in C$ in the graph $G_{lc}$. We can then define loose concepts as a function \(lc : G \rightarrow \mathbb{N}_{0}\), with \[lc(g) = \left|\left\{c \in C : deg(c) = 0\right\}\right|\]
% \end{definition}

\subsubsection{Weakly Connected Components} indicate that the vocabulary is split into separate ``clusters''. Presence of several WCC might be caused by incomplete data acquisition, ``forgotten'' test data, outdated terms, accidental deletion of relations and the like. In a practical setting, existence of weakly connected components could render the vocabulary less suitable for operations that rely on navigating a connected vocabulary structure, such as query expansion or suggestion of related terms. To calculate WCC, we create an undirected graph whose set of nodes constitutes all non-loose concepts and whose edges are defined by $SR$. We then utilize ``Tarjan’s algorithm''\cite{Hopcroft1973} that finds all connected components of the graph, i.e. all sets of conceptual resources that can reach each other by a path of semantic relations.

% \begin{definition}
% To evaluate this criterion we consider a graph \(G_{wcc} = (C - lc(G_{lc}), SR)\). The function \(components(g) : G \rightarrow  \mathfrak{P}(G)\) calculates all weakly connected components of a graph by replacing all directed edges by undirected edges and utilizing “Tarjan’s algorithm” \cite{Hopcroft1973}. Based on that function, we can express this criterion as a function \(wcc : G \rightarrow \mathbb{N}_{0}\), with \[wcc(g) = \left|components(g)\right|\]
% \end{definition}
 

\subsubsection{Cyclic Hierarchical Relations} have been mentioned numerous times in thesaurus development literature. \cite{Soergel2002} suggests a ``check for hierarchy cycles" since they ``throw the program for a loop in the generation of a complete hierarchical structure''. However, SKOS does not define a formal constraint regarding hierarchy cycles but the SKOS reference document mentions them as potential problems. The SKOS documentation also does not state, how exactly hierarchical relationships are to be interpreted. There exist common forms like, e.g., ``generic-specific'', ``instance-of'' or ``whole-part'' \cite{Hedden2010,Harpring2010,Aitchison2000} where cycles would be considered a logical contradiction. To calculate this criterion, we construct two graphs, with the set of nodes being $C$. The edges in one graph only point towards the broader concepts, the edges of the other graph towards the narrower concepts. For each graph we now identify those nodes that are part of a cycle and return the set of strongly connected components they are contained in.

%Erkärung: ein knoten (konzept) liegt oft in vielen überlappenden kreisen und es würde keinen sinn machen, diese alle auszugeben. deshalb berechne ich mir von jedem knoten der in einem kreis liegt, in welcher starken zusammenhangskomponente er liegt. das sind dann per definition alle knoten, die mit irgendeinem kreis in dem das konzept liegt was zu tun haben, weil eine starke zusammenhangskomponente ja eigentlich nur aus kreisen besteht (von jedem knoten gibt es einen gerichteten weg zu jedem anderen knoten)
 
% \begin{definition}
% Let \(H_{br} \subseteq SR\) be the set of pairs of hierarchically broader related resources, i.e. \(I_{EXT}(\texttt{skos:broader}^\mathcal{I}) \cup I_{EXT}(\texttt{skos:broaderTransitive}^\mathcal{I}) \cup I_{EXT}(\texttt{skos:broadMatch}^\mathcal{I})\). \(H_{br}\) also includes the extensions of the respecitive inverse properties with each pair in reverse order. Likewise, \(H_{nar} \subseteq SR\) can be defined as the set of hierarchically narrower related resources. We consider two graphs \(G_{br} = (C,H_{br})\) and \(G_{nar}=(C,H_{nar})\) and a function \(cycleNodes:G \rightarrow \mathfrak{P}(C)\) identifying all nodes in a graph, that are part of a cycle. Furthermore we consider a function \(stronglyConnectedSets: G \rightarrow \mathfrak{P}(C)\) that calculates all sets of nodes contained in a strongly connected subgraph of $G$. This criterion can be defined as a function \(chr:G \rightarrow \mathbb{N}_{0}\) with \[chr(g)=\left|\left\{s \in stronglyConnectedSets(g) : cn \in cycleNodes(g), cn \in s\right\}\right|\]
% \end{definition}


\subsubsection{Valueless Associative Relations}
The ISO/DIS 25964-1 standard suggests that terms that share a common broader term but don’t have an overlapping meaning, should not be related associatively. This is also advocated by \cite{Hedden2010} and \cite{Aitchison2000} who also mentions ``the risk that thesaurus compilers may overload the thesaurus with valueless relationships'', having a negative effect on precision. This criterion identifies a set of pairs of conceptual resources that share the same broader or narrower concept while also being associatively related, i.e. all pairs in $I_{EXT}(\texttt{skos:related}^\mathcal{I})$ that contain both of these conceptual resources.

\subsubsection{Solely Transitively Related Concepts}
Two concepts are related using only transitive hierarchical relations which are, according to the SKOS reference document, ``not used to make assertions''. Transitive hierarchical relations in SKOS are meant to be infered by the vocabulary user, which is reflected in the SKOS schema by, e.g., \texttt{broader} being a subproperty of \texttt{broaderTransitive}. This criterion identifies the set $I_{EXT}(\texttt{skos:broaderTransitive}^\mathcal{I}) \cup I_{EXT}(\texttt{skos:narrowerTransitive}^\mathcal{I})$ \textbf{without} RDFS subproperty entailment.

\subsubsection{Omitted Top Concepts}
The SKOS language provides ConceptSchemes which are a facility for grouping related concepts. In order to provide entry points to such a group of concepts, one or more concepts can be marked as top concepts. This helps to provide ``efficient access'' (SKOS primer) and simplifies orientation in the vocabulary. To calculate this criterion, we identify all ConceptSchemes $I_{CEXT}(\texttt{skos:ConceptScheme}^\mathcal{I})$ not contained in any of the pairs in $I_{EXT}(\texttt{skos:hasTopConcept}^\mathcal{I})$.

%Thus, this criterion returns the subset of $C$ whose elements are not contained in any of the pairs obtained by $I_{EXT}(\texttt{skos:hasTopConcept}^\mathcal{I})$.

\subsubsection{Top Concept Having Broader Concepts}
\cite{Allemang2011} proposes to ``not indicate any concepts internal to the tree as top concepts'', i. e. top concepts should not have broader concepts. This criteria identifies all elements from $C$ that occur in both $I_{EXT}(\texttt{skos:topConceptOf}^\mathcal{I})$ and as first element of each pair from $I_{EXT}(\texttt{skos:broader}^\mathcal{I})$.

% LINKED-DATA SPECIFIC CRITERIA

\subsection{Linked Data Specific Criteria}

This section identifies criteria related to a vocabulary's interaction with other datasets in the LOD cloud.

\subsubsection{Missing In-Links}
A periodically updated version of a diagram visualizing ``the data sets in the LOD cloud as well as their interlinkage relationships'' is publised online\footnote{\url{http://www4.wiwiss.fu-berlin.de/lodcloud/state/}}. In these diagrams, dbpedia is shown as the most intensely linked resource, i.e. a large number of vocabularies link to resources in that dataset. Therefore a dataset with a large number of other datasets referencing it, can be assumed to be of great value for the community. Accordingly, estimating the number of In-Links of SKOS concepts defined in a vocabulary (henceforth called the in-degree of a concept) would give some hint about how established or accepted a controlled vocabulary is. A method to carry out these estimations would be utilizing existing Linked Data indices like, e.g., Sindice\footnote{\url{http://sindice.com/}} that provide a SPARQL endpoint. In order to estimate the number of In-Links for a conceptual resource, we iterate over all elements in $C$ and query the Sindice SPARQL endpoint for triples containing the concept's URI in the object part. Those concepts having an empty query result, are missing any In-Links and are thus included in the result set of this criterion.

\subsubsection{Missing Out-Links}
The Web of Data consists of globally interconnected datasets, ``enabling seamless connections between data sets''\cite{Heath2011}. These links can be established between various SKOS vocabularies by Out-Links, i.e., references from conceptual resources in the ``source vocabulary'' $V$ to resources identified by an HTTP URI and available at a different host. Similar to the criterion above, this criterion identifies the set of all conceptual resources that lack such Out-Links. Calculation of the result set is done by iterating over all elements $c \in C$ and investigating the host part of all related resources. If none of these host names differs from the host part of $c$, it has no Out-Links and is thus included into the resources identified by this criterion.

%\subsubsection{HTTP URI Schema Violation}\todo{All}{We don't have any evidence this really happens in existing vocabularies, although it is very likely. Maybe we should skip this.}

\subsubsection{Link Target Unavailability}
This criterion identifies a set of all resources that have been identified as unavailable, i.e. resolving their URI leads to an erroneous HTTP response or no response at all. An erroneous HTTP response in that case can be defined as a response code (after a possible redirection) other than 200. Just as in the ``traditional'' Web, these ``broken links'' hinder information gathering and should be avoided. Calculation of this criterion is performed by iterating over all resources in $IR$, performing an HTTP request and include those resources in the result set that are unavailable.

% OTHER CRITERIA

\subsection{Other Criteria}

\subsubsection{Undefined SKOS Resource Usage}
Some vocabularies reference resources in the SKOS namespace that cannot be resolved. A common reason for this is, that there seems to exist a misunderstanding about the correct way of introducing new classes or properties in an OWL document. In some cases, vocabularies ``invent'' new terms in the SKOS namespace in order to meet certain requirements of the publishing organizations that cannot (yet) be expressed within the SKOS schema. Another reason for the presence of unresolvable resource references in the SKOS namespace is that the vocabulary might be outdated. With the SKOS W3C recommendation maturing, some properties have been removed from the current version of the schema. Vocabularies using these ``deprecated'' properties hence contain parts of their information in a non ``standard"-compliant way, having negative effects when, e.g., crawling vocabularies to gather specific information. A method to identify these undefined SKOS resources is to iterate over all resources in $IR$, including (i) those resources in the resulting set that are contained in the list of deprecated resources or (ii) have an URI in the SKOS namespace but are not defined in the SKOS ontology. 



%!TEX root = main.tex

\section{Analysis of Existing SKOS Vocabularies}\label{sec:analysis}

To learn about possible quality issues in real-world vocabularies, we implemented the previously described quality checking function in the qSKOS tool and applied each function on a set of existing SKOS vocabularies. 

\subsection{The qSKOS Quality Checking Tool}

The open-source qSKOS\footnote{\url{https://github.com/cmader/qSKOS/}} quality assessment tool can be used to find possible quality issues a given SKOS vocabulary. Users can run the tool by passing a vocabulary and selecting the quality checking functions they want to perform. As a result, they obtain detailed reports listing possibly affected resources alongside with the potential cause of the issue. qSKOS is implemented in Java and can be used as standalone command-line tool or API in any other application. Its design is open for the introduction of additional quality functions. In our analysis, we used qSKOS version 0.2., which can be downloaded from: \url{https://github.com/cmader/qSKOS/releases/qSKOS-0.2.tar.gz}.

\todo{BH}{I think we can remove the collections column, none of the issues involve collections}

\subsection{Vocabulary Data Set}

We used Sindice and available listings\footnote{\url{http://www.w3.org/2001/sw/wiki/SKOS/Datasets}} to learn about existing SKOS vocabularies. Table~\ref{tab:vocabs} lists our representative vocabulary selection and shows that the vocabularies differ in size and application domain: the \emph{Gemeenschappelijke Thesaurus Audiovisuele Archieven (GTAA)} is vocabulary from the media domain, \emph{Eurovoc} and the \emph{Integrated Public Sector Vocabularies (PSV)} cover public sector concepts, the \emph{Medical Subject Headings (MeSH)} and \emph{Peroxisome Knowledge Base (PXV)} are from the (bio-)medical domain, the \emph{Geonames Ontology} covers locations, the \emph{Thesaurus for Economics (STW)} and \emph{North American Industry Classification System (NAICS)} are vocabularies from the economics and business area and the \emph{DBPedia Categories} reflect Wikipedia's user-generated categorization system. We also include the \emph{Meketre} vocabulary, which defines Egyptology-related concepts, and the non-public \emph{Austrian Armed Forces Thesaurus (LVAk)}, which we generated from CSV files provided by the vocabulary maintainers. A zip-file containing all 14 public vocabulariesx we used in our analysis, can be downloaded from: \url{http://www.github.com/cmader/???}. In contrast to the column ``Concepts'' that shows the total number of SKOS Concepts that can be infered from the vocabulary, the column ``Authoritative Concepts`` lists only those concepts that have an URI in the namespace of the vocabulary. These concepts are of importance when finding Out-Links to other vocabularies on the LOD cloud.

\begin{table}
\label{tab:vocabs}
\caption{Analyzed SKOS vocabularies}
    
\begin{center}
\resizebox{\textwidth}{!} {
\setlength{\extrarowheight}{5pt}

\begin{tabular}{p{6cm}ccccccccc}

\textbf{Vocabulary} & \rotatebox{90}{\textbf{Abbreviation}} & \rotatebox{90}{\textbf{Version/last mod.}} & \rotatebox{90}{\textbf{Concepts}} & \rotatebox{90}{\textbf{Auth. Concepts}} & \rotatebox{90}{\textbf{Labels}} & \rotatebox{90}{\textbf{Semantic Rel.}} & \rotatebox{90}{\textbf{Aggregation Rel.}} & \rotatebox{90}{\textbf{Concept Schemes}} & \rotatebox{90}{\textbf{Collections}}\\
\toprule
Agricultural Thesaurus & \textbf{AGROVOC} & 1.3 & 32,035 & 32,035 & 620,629 & 65,934 & 32,085 & 1 & 0 \\
\hline
DBpedia Categories & \textbf{DBpedia} & ? & 743,410 & 743,410 & 740,352 & 1,490,316 & 0 & 0 & 0 \\
\hline
The EU's multilingual thesaurus & \textbf{Eurovoc} & 5.0 & 6,797 & 6,797 & 457,788 & 18,491 & 15,512 & 128 & 0 \\
\hline
Geonames Ontology & \textbf{Geonames} & 2.2.1 & 671 & 671 & 671 & 0 & 671 & 9 & 0 \\
\hline
Gemeenschappelijke Thesaurus Audiovisuele Archieven & \textbf{GTAA} & 2010/08/25 & 171,991 & 171,991 & 178,776 & 50,892 & 343,980 & 9 & 0 \\
\hline
Integrated Public Sector Vocabulary & \textbf{IPSV} & 2.00 & 4,732 && 7,945 & 13,843 & 4,483 & 3 & 0 \\
\hline
Library of Congress Subject Headings & \textbf{LCSH} & 2011/08/09 & 459,182 & 407,908 & 746,076 & 595,754 & 815,816 & 19 & 0 \\
\hline
Austrian Armed Forces Thesaurus & \textbf{LVAk} & n/a & 13,411 & 13,411 & 17,250 & 16,346 & 0 & 0 & 0 \\
\hline
Middle Kingdom tombs of Ancient Egypt Thesaurus & \textbf{Meketre} & 2011/07/07 & 422 & 422 & 569 & 1,698 & 6 & 2 & 0 \\
\hline
Medical Subject Headings & \textbf{MeSH} & ? & 24,626 & 24,626 & 150,617 & 38,858 & 0 & 0 & 0 \\
\hline
North American Industry Classification System & \textbf{NAICS} & 2012 & 4,175 & 2,213 & 0 & 8,684 & 2,235 & 1 & 0 \\
\hline
New York Times People & \textbf{NYTP} & 2010/06/22 & 4,979 & 4,979 & 4,979 & 0 & 4,979 & 1 & 0 \\
\hline
University of Southampton Pressinfo & \textbf{Pressinfo} & 2011/02/24 & 1,125 & 1,125 & 0 & 0 & 0 & 0 & 0 \\
\hline
Peroxisome Knowledge Base & \textbf{PXV} & 1.6 & 2,112 & 1,686 & 3,628 & 2,695 & 1,716 & 1 & 0 \\
\hline
Thesaurus for Economics & \textbf{STW} & 8.06 & 6,524 & 6,524 & 31,189 & 57,907 & 6,531 & 1 & 0 \\
\bottomrule
\end{tabular}

}
\end{center}
\end{table}

\subsection{Results}

The results of this analysis are reported in Table~\ref{tab:results}, which shows the absolute number of possibly affected resources for each quality checking function and vocabulary combination. The value 0 means that we haven't found any affected resource, N/A means that a certain function was not applicable for reasons we explain below. An asterisk after certain numeric values indicates that due to performance reason, only an extrapolated subset containing 5\% of the respective elements (HTTP URIs or concepts) of the vocabulary has been analyzed.

\begin{table}[h]
\label{tab:results}
\caption{Results of the quality checking functions}

\begin{center}
\resizebox{\textwidth}{!} {
\setlength{\extrarowheight}{5pt}

\begin{tabular}{p{4cm}ccccccccccccccc}
\textbf{Issue} & \rotatebox{90}{\textbf{GTAA}} & \rotatebox{90}{\textbf{Geonames}} & \rotatebox{90}{\textbf{MeSH}} & \rotatebox{90}{\textbf{PXV}} & \rotatebox{90}{\textbf{Eurovoc}} & \rotatebox{90}{\textbf{IPSV}} & \rotatebox{90}{\textbf{Agrovoc}} & \rotatebox{90}{\textbf{DBpedia}} & \rotatebox{90}{\textbf{Pressinfo}} & \rotatebox{90}{\textbf{NYTP}} & \rotatebox{90}{\textbf{LCSH}} & \rotatebox{90}{\textbf{Meketre}} & \rotatebox{90}{\textbf{STW}} & \rotatebox{90}{\textbf{NAICS}} & \rotatebox{90}{\textbf{LVAk}} \\
\toprule
Omitted or Invalid Language Tags & 0 & 0 & 23,950 & 1,578 & n/a & 0 & 0 & 0 & 1,224 & 0 & 18 & 0 & 2 & n/a & 13,411 \\

Incomplete Language Coverage & 0 & 0 & 0 & 0 & n/a & 0 & 32,035 & 0 & 0 & 0 & 0 & 420 & 6,456 & n/a & 0 \\

Undocumented Concepts & 96,850 & 0 & 1,807 & 1,918 & 5,341 & 4,551 & 32,035 & 743,410 & 1,125 & 4,094 & 398,036 & 422 & 5,236 & 3,259 & 13,411 \\

Label Conflicts & 12,404 & 18 & 0 & 7 & n/a & 0 & 2,949 & 0 & 0 & 0 & tba & 4 & 5 & n/a & 13 \\

\midrule

Orphan Concepts & 162,000 & 671 & 0 & 2 & 7 & 0 & 0 & 77,062 & 1,125 & 4,979 & 172,364 & 0 & 4 & 0 & 21 \\

Weakly Connected Components & 621 & 0 & 4 & 10 & 4 & 1 & 4 & 1,506 & 0 & 0 & 22,131 & 5 & 1 & 1 & 11 \\

Cyclic Hierarchical Relations & 0 & 0 & 4 & 0 & 0 & 0 & 0 & 1,132 & 0 & 0 & 0 & 0 & 0 & 0 & 5 \\

Valueless Associative Relations & 9,438 & 0 & 495 & 0 & 1 && 282 & 8,120 & 0 & 0 & 1,879 & 0 & 5,082 & 0 & 5 \\

Solely Transitively Related Concepts & 0 & 0 & 0 & 0 & 2,652 && 0 & 0 & 0 & 0 & 0 & 36 & 0 & 2,189 & 0 \\

Omitted Top Concepts & 9 & 9 & 0 & 0 & 1 && 0 & 0 & 0 & 1 & 18 & 0 & 0 & 0 & 0 \\

Top Concept Having Broader Concepts & 0 & 0 & 0 & 1 & 0 && 0 & 0 & 0 & 0 & 0 & 0 & 0 & 0 & 0 \\

\midrule

Missing In-Links && 19 && 1,686 &&&&& 1,125 & 20 && 422 & 6,516 & 2,213 & 13,411 \\

Missing Out-Links & 171,991 & 671 && 1,472 & 6,797 && 32,035 & 743,410 & 1,116 & 0 & 408,198 & 273 & 6,524 & 0 & 13,411 \\

Broken Links & 0 & 0 & 1 & 163 &&& 238 & 0* & 11 & 7 && 425 & 1 & 3,169 & n/a \\


Undefined SKOS Resources & 0 & 0 & 1 & 0 & 0 & 1 & 0 & 0 & 0 & 0 & 0 & 0 & 0 & 0 & 0  \\

\bottomrule
\end{tabular}
}
\end{center}
\end{table}

% Labeling and Documentation Issues
We found labeling and documentation issues in all analyzed vocabularies. PXV provides labels for XX\% of its concepts, Pressinfo labels only for 1 out of 1125 concepts, and the STW thesaurus misses labels for two concepts. Eurovoc and NAICS use SKOS-XL labels, which are currently not supported by qSKOS.
Incomplete Language Coverage were issues in the three analyzed vocabularies: in AGROVOC all concepts had incomplete coverage, probably because \todo{CM}{es gibt labels in insgesamt 25 versch sprachen und keines der konzepte ist in jeder sprache beschrieben}, and in STW we found XX\%.
Most vocabularies didn't document their concepts with the documentary properties defined by SKOS.
Label conflicts, hence \todo{CM}{explain briefly} were found in several vocabularies: GTAA has many because? PXV has 7. Why given examples?

% Structural Issues


% Linked Data Specific Issues


% The \textbf{Loose Concepts} metrics provides insights on the structural complexity of vocabularies. The large number of loose concepts in GTAA can be explained by the fact that it contains names (e.g., persons or places) that arent't connected using skos properties. This is also the case for Geonames, NYTP and Pressinfo where the number of total conceps is equal to the number of loose concepts, indicating that no concept is semantically related to another concept.
% 
% The large number of \textbf{Weakly Connected Components} in GTAA is caused by many ``minimal'' components, containing only 2 concepts connected by a \texttt{skos:related} property. Due to the fact that loose concepts are not counted as weakly connected components, Geonames, NYTP and Pressinfo evaluate to zero components.


%!TEX root = main.tex
\section{Related Work}\label{sec:related_work}

The problem of ``vocabulary quality'' is closely related to the more general problem of “data quality”, which has intensively been discussed in data and information systems research. \cite{Pipino2002,Batini2009} argue that dealing with data quality should involve both “subjective perceptions of the individuals” and “objective measurements based on the data set in question”. We see our work as a contribution to the latter and believe that the results of SKOS quality measures must be combined with domain knowledge, and therefore human expertise, to make final quality judgements.

A vast number of publications in the field of designing, validating and testing controlled vocabularies is available. Many of them propose properties \cite{Soergel1995}, design guidelines \cite{Svenonius2003,...} or metrics \cite{Elkin2002,Kless2010} aiming to improve, e.g., retrieval precision and recall, consistency and multilinguality. The work is based on various sources like, e.g., review of existing vocabularies~\cite{Soergel1995} or survey-based studies~\cite{Pinto2008}. However, there are hardly any formally defined quality indicators that can be automatically assessed without further knowledge about the application domain, targeted user group or usage scenario.

%For controlled vocabularies, \cite{Svenonius2003} argues precision and recall to be “the chief objectives of any retrieval language” and proposes a set of design guidelines (e.g., term selection, structural and syntactical considerations) aiming to improve these objectives. Elkin and XXX? [Elkin2002] identify general quality metrics and structural and maintainability considerations for controlled vocabularies in the health domain, which cover issues like non-redundancy, consistency, and multilinguality of concepts? Soergel [Soergel1995] mentions properties of “good” thesauri like, e.g., support for synonym and hierarchic expansion or request-oriented indexing. As negative properties of the AAT Soergel points out, e.g., missing cross-references, incomplete facet analysis, the monohierarchical layout and shortcomings in term form and choice. On the positive side he mentions, e.g., presence of scope notes, definition of related terms and some definitions of synonyms. Pinto [Pinto2008] performs a survey-based study comparing perceived and expected properties (“variables”) of thesauri used in social science databases. The findings exposed need for “considerable improvement” in the structural perspective (e.g., equivalence and associative relationships), but also regarding “performance” variables (e.g., explanatory notes or expected performance) and “format” issues (e.g., ergonomics and display). Some of the identified variables (e.g, pre-coordination, searching options) are hard do assess automatically or are application dependent, others aren’t exactly defined or formalized. However, neither of them suggest any automatic assessment methods, which can be applied to controlled Web vocabularies.

<<<<<<< HEAD
Research on the notion of “data quality” has also been conducted in Semantic Web research. For general datasets, \cite{Heath2011,Hogan2010} propose best practices and ..   \cite{Tartir2007} 

Hogan and ??? [Hogan2009] identify four categories of common errors and shortcomings in RDF documents. Some of them are also of great importance and can be adopted for assessing the quality of SKOS vocabularies, especially considering interoperability issues like e.g., dereferencability and use of undefined classes and properties. However, the author’s focus lies on RDF datasets in general, thus features of controlled vocabularies are not covered. This is also the case for [Tartir2007] who propose a framework for evaluation and ranking of Ontologies. Given the fact that vocabularies usually extend the SKOS schema only to a very limited degree by defining subclasses and subproperties, the metrics defined in [Tartir2007] are not considered applicable to these vocabularies.
=======
Research on the notion of “data quality” has also been conducted in Semantic Web research. Hogan and ??? [Hogan2009] identify four categories of common errors and shortcomings in RDF documents. Some of them are also of great importance and can be adopted for assessing the quality of SKOS vocabularies, especially considering interoperability issues like e.g., dereferencability and use of undefined classes and properties. However, the author’s focus lies on RDF datasets in general, thus features of controlled vocabularies are not covered. This is also the case for [Tartir2007] who propose a framework for evaluation and ranking of Ontologies. Given the fact that vocabularies usually extend the SKOS schema only to a very limited degree by defining subclasses and subproperties, the metrics defined in [Tartir2007] are not considered applicable to these vocabularies.
>>>>>>> 2a5aecf06681ee581815e0968fdf0401784682ee

[Arpinar2006] define three types of conflicts that may occur in ontologies. Based on these types, the ontology maintainers can create rules in RuleML or SWRL to find violations in their ontologies. Although not using a rule language, we express quality criteria formally, following a very similar approach.

[Gangemi2005] introduce three “measure types for ontology evaluation” (dimensions), i.e. structural, functional and usability-related measures. The structural exploits the graph structure of an ontology and defines a number of measure like, e.g., absolute, average or maximum depth and breadth, number of leaf nodes, number of siblings or density. While for many of the measures, its applicability for controlled vocabularies is not immediately clear, some (e.g., cycle ratio, inverse relations) resemble common patterns in controlled vocabularies and thus are represented in our catalog of quality criteria. Functional dimensions are measures related to how well an ontology fits its intended purpose like, e.g., recall, precision and accuracy. They are highly context-dependent, however, we expect most of our identified criteria to impact these dimensons. In the usability dimension, Gangemi et al. mention three analytical levels for usability profiling: recognition, efficiency, and interfacing. Since they don’t go into details on what the implications for thesauri are, we try to complement this in the rationale of our identified quality crtieria. [Brank2005] provide a concise overview on ontology evaluation approaches
Linked Data Quality

[Yeganeh2011] propose a method to convert semi-structured data (e.g., XML files) into “high-quality Linked Data”. Their notion of quality encompasses usage of HTTP URIs, URI dereferencability, linkage of related objects as well as merging and detection of duplicate resources.


%!TEX root = main.tex
\section{Conclusions and Future Work}\label{sec:conclusions}

% make clear that this shouldn't replace taxonomists

% web app

% already improvements in two vocabularies

\todo{ALL}{discuss implications of OWA on SKOS quality.}


% We skip this for the submission:

% \section{Acknowledgements}
% Tom Dent (esd support team) for providing a dump for the IPSV vocabulary
% Andrew Gibson for providing a dump for the Peroxisome knowledge Base and giving valuable feedback on their vocabulary structure.


\bibliography{references}
\bibliographystyle{splncs03}

\end{document}

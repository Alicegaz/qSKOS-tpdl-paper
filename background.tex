%!TEX root = main.tex

\section{Background and Methodology}

% About Controlled Vocabulary Quality

\todo{CM}{After the introduction it is very common to introduce the problem in more detail. We should write a page about quality in controlled vocabularies and explain the sources for our quality metics. Maybe you can pull parts from the related work section.}

% Methodology

\todo{CM}{How did we identify the criteria we present in the following section?}

%SKOS is a language for defining vocabularies in the Web of Data and therefore based on the Open World Assumption. Established quality notions from closed-world systems, such as referential integrity or schema validation, don’t hold anymore, because the available information may be incomplete and facts that are not explicitly stated cannot be determined as true or false. While trust and provenance models for Web data are being developed \cite{Omitola2011,Hartig2009}, content-based and hand-crafted heuristics are currently used to evaluate quality in Linked Data sets \cite{Heath2011}.
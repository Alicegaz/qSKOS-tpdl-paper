%!TEX root = main.tex

\section{Background}\label{sec:background}

% About Quality in SKOS vocabularies

SKOS is a language for defining vocabularies in the Web of Data and therefore based on the Open World Assumption. Established quality notions from closed-world systems, such as referential integrity or schema validation, don't hold anymore, because available information may be incomplete and non-explicitly stated facts cannot be determined as true or false. While trust and provenance models for Web data are being developed~\cite{Omitola2011,Hartig2009}, content-based and hand-crafted heuristics are currently used to evaluate quality in Linked Data sets~\cite{Heath2011}.

% MOVE TO RELATED WORK: Gueret et al. [\todo{CM}{find the right reference in google scholar}], for instance, calculate network measures to support human judgement on the quality of Linked Data \todo{CM}{?} graphs.


%In the course of the LATC project, Gueret et al. propose an infrastructure that utilizes crowd-sourcing technology to semi-automatically create links between datasets of the LOD cloud. To ensure the quality of the resulting network, human judgement is supported by calculating a report of locally approximated network measures. \todo{BH}{ich denke antoine hat das paper gemeint: http://www.mendeley.com/download/public/18928/4084690763/56bdde328dfbba2cb8594b57152c0e132ef00a40/dl.pdf . Leider fehlen da etliche Details um sich ein konkretes Bild machen zu können}

% SKOS Integrity Conditions (http://www.w3.org/TR/skos-reference/):
% S9    skos:ConceptScheme is disjoint with skos:Concept.
% S13   skos:prefLabel, skos:altLabel and skos:hiddenLabel are pairwise disjoint properties. (* NOT expressed formally *)
% S14   A resource has no more than one value of skos:prefLabel per language tag. (* NOT expressed formally *)
% S27   skos:related is disjoint with the property skos:broaderTransitive. (* NOT expressed formally *)
% S37   skos:Collection is disjoint with each of skos:Concept and skos:ConceptScheme.
% S46   skos:exactMatch is disjoint with each of the properties skos:broadMatch and skos:relatedMatch. (* NOT expressed formally *)
% -> 2 of 6 are expressed formally in OWL; the others cannot be expressed because OWL doesn't support disjoint properties

% Poolparty quality checker implementation:
% S13, S14, S27, S46 plus the formally defined S9, S37
% syntactic checks: URI, character encoding, whitespaces
% missing language tags: in SKOS labels and textual content)
% missing labels: prefLabels for skos:Concepts and rdfs:labels for skos:ConceptSchemes)
% loose concepts: concepts that are no top concepts and have no broaders

% SKOSify
% S13, S14, S27

% Mapping qSKOS Integrity Conditions <-> qSKOS quality criteria
% S13 + S14 combined in "Ambigious labels"
% S27 + S46 combined in "Associative vs. Hierarchical Relation Clashes" and "Exact vs. Associative and Hierarchical Mapping Clashes"

Interoperability between applications has been the major motivation for specifying in total six \textbf{SKOS integrity conditions}~\cite{Miles2005}, each of which is a statement that defines under which circumstances data are consistent with the SKOS data model. An example condition is ``a resource has no more than one value of \texttt{skos:prefLabel} per language tag''. Tools that can check whether these integrity conditions are met for given data are already available: two of the six integrity conditions are defined formally in the OWL representation of SKOS and can therefore be validated by any OWL reasoner. For validating a SKOS vocabulary against the integrity conditions, one can use tools such as the online \textbf{PoolParty Thesaurus Consistency Checker}, or the \textbf{Skosify} command line utility, which can validate SKOS vocabularies and also correct some detected quality problems.

% Argue why we need criteria that go beyond low-level integrity constraints and why we need to formalize them

Typical application areas of controlled vocabularies are classification, indexing, autocompletion, query reformulation/expansion, or serving as a glossary. As we discussed in detail in our earlier work \cite{Nagy2011}, these application areas have specific requirements with respect to vocabulary features, such as structure, availability, and documentation of conceptual resources. A vocabulary that doesn't fulfill these requirements can have a negative effect on an application's output and would therefore be judged by the application designers and potentially also by users as being a low-quality vocabulary. Missing relations between conceptual resources, for instance, can reduce retrieval recall when a vocabulary is used for query expansion. Missing entry points to concepts in a vocabulary can impede orientation for human users. Proprietary term definitions can hinder queries across distinct datasets, to name some examples.


% Argue why we need to formalize them -> because otherwise they will be implemented inconsistently across applications
Thus, in order to help vocabulary and application designers in judging and improving the quality of SKOS vocabularies we must identify possible quality issues that go beyond integrity conditions defining data model consistency. We can identify such issues by examining quality criteria for controlled vocabularies that have already been discussed in standardized guidelines~\cite{ISO25964-1:2011,Z39.19:2005}, manuals~\cite{Svenonius2003,Hedden2010,Aitchison2000,Harpring2010}, case studies~\cite{Soergel1995}, and scholarly articles \cite{Coronado2009,Soergel1997,Soergel2002,Elkin2002,Kless2010}. Since we want to provide tool support for quality checking, we must formalize the identified issues into computable quality checking functions taking into account the syntax, structure and semantics of SKOS. This also ensures consistent implementation across tools.


% TODO ADD TO THIS SECTION: a vast number of publications in the field of designing, validating and testing controlled vocabularies is available. Many of them propose properties \cite{Soergel1995}, design guidelines \cite{Svenonius2003,...} or metrics \cite{Elkin2002,Kless2010} aiming to improve, e.g., retrieval precision and recall, consistency and multilinguality. The work is based on various sources like, e.g., review of existing vocabularies~\cite{Soergel1995} or survey-based studies~\cite{Pinto2008}. However, there are hardly any formally defined quality indicators that can be automatically assessed without further knowledge about the application domain, targeted user group or usage scenario.



%Whilst aformentioned literature also deals with online vocabulary publication and interoperability \cite{ISO25964-1:2011,Hedden2010}, covering requirements for publication on the Web of Data (as in, e.g., \cite{Heath2011,Hogan2010,Allemang2011}) goes beyond their scope. 
%On the other hand, publications targeting dataset and ontology development for the Web of Data don't elaborate on implications of the design decisions for controlled vocabularies. Our work seeks to narrow this gap by...

% TODO: bitte hier einfach nur die referenzen auf die wir uns beziehen zusammenfassen und kategorisieren. Schau dir dazu auch nochmals die SKOSify referenzen und die aus eurem iSemantics paper an.



% I think for the big picture it is not necessary to go into such details. We can do this later when introducting the metrics

% The PoolParty Thesaurus Consistency Checker implements all these conditions and introduces the following new checks: \textbf{URI validation} checks for invalid characters like, e.g. whitespaces in the URI string. \textbf{Definition of labels} is tested for conceptual resources and concept schemes. Furthermore, SKOS labels and textual content is checked for \textbf{missing language tags} and the notion of \textbf{loose concepts} is defined: it encompasses conceptual resources that are no top concepts and have no broader concept.
% 
% The SKOSify tool implements three of the above mentioned SKOS integrity conditions as well as the \textbf{missing language tags} and \textbf{loose concepts} measures introduced by the PoolParty checker. It introduces \textbf{hierarchy cycle detection} and detection of \textbf{extra whitespace} for SKOS label or documentation properties.

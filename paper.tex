\documentclass{llncs}

\usepackage{verbatim}
\usepackage{url}
\usepackage{amssymb}

\title{Quality Criteria for SKOS Vocabularies}
\author{Christian Mader\inst{1} \and Bernhard Haslhofer\inst{2}}
\institute{
	University of Vienna, Faculty of Computer Science\\\email{christian.mader@univie.ac.at}
	\and Cornell University, Information Science\\\email{bernhard.haslhofer@cornell.edu}}

\begin{document}

\maketitle

\begin{abstract}
\end{abstract}

\section{Introduction}
The Simple Knowledge Organization System (SKOS\footnote{Simple Knowledge Organization System, \url{http://www.w3.org/2004/02/skos/}}) has become a de-facto standard for expressing controlled vocabularies on the Web. Many organizations, including the European Union\footnote{EuroVoc, \url{http://eurovoc.europa.eu/}}, the United Nations\footnote{AGROVOC, \url{http://aims.fao.org/website/AGROVOC-Thesaurus/sub}}, or the UK government\footnote{Integrated Public Sector Vocabulary (IPSV), \url{http://doc.esd.org.uk/IPSV}} have created SKOS representations of their controlled vocabularies and published them on the Web so that they can easily be navigated and reused by humans and, most importantly, also by applications. Each vocabulary is the result of a specific design process that reflects domain-specific requirements as well as the target audience the vocabulary is created for. Orthogonal to the design process, vocabulary \textbf{designers and maintainers} usually apply existing guidelines or follow their own rules when defining concepts, in order to meet their \textbf{quality requirements} as, for instance, described in \cite{Coronado2009}. As a consequence, currently available vocabularies are, even though they are expressed in SKOS, often heterogeneous in terms of quality, which can affect their interoperability and applicability for tasks like query expansion, faceted browsing, clustering, or auto-complete suggestions.

The following representative examples, which we identified by looking at real-world SKOS vocabularies, illustrate the \textbf{quality problems} applications could face when making use of them:
\begin{itemize}
	\item The SKOS version of Agrovoc defines concepts in 25 different languages. However, almost all concepts have an English description, 38\% of all concepts don’t have German labels attached. This can be a problem for applications that use Agrovoc and rely on German translations.

	\item The previous version of the STW thesaurus, contained 5 pairs of concepts with identical labels. As a result, the auto-complete function of the online search interface suggested identical entries, with no disambiguation information.

	\item The vocabulary developed by the Austrian Armed Forces (LVAk thesaurus\footnote{Non-public thesaurus developed by the Austrian Armed Forces}), which contains 11 disconnected concept clusters. Confronting the thesaurus maintainers with these structures, they recognized them as “forgotten” test data that has no practical significance.

	\item In the DBpedia category\footnote{\url{http://downloads.dbpedia.org/3.7/en/skos_categories_en.nt.bz2}} hierarchy, more than 10 percent of the categories not linked to any other category. When other vocabularies link to the DBPedia categories, e.g., to enable query expansion, it can negatively affect recall and reduce navigability.
\end{itemize}

One reason for these quality problems is that SKOS currently only defines a set of \textbf{low-level integrity conditions}, which, when being enforced, ensure the structural and semantic requirements of the SKOS specifications but fail to capture \textbf{higher-level quality issues}, as those mentioned in the examples before. Another problem is that existing vocabulary design guidelines and best practices purely rely on \textbf{human assessment}, which is highly subjective and doesn’t scale when vocabulary quality assessment should be performed on larger, possibly linked Web vocabularies.

We believe, and recent developments of quality checking tools (e.g., PoolParty Thesaurus Consistency Checker\footnote{\url{http://demo.semantic-web.at:8080/SkosServices/check}}, qSKOS\footnote{\url{https://github.com/cmader/qSKOS}}, SKOSify\footnote{\url{http://code.google.com/p/skosify/}}) support this motivation. This, however, requires a formalization of the notion of quality with respect to SKOS vocabularies, which can be used in order to consistently implement quality assessment procedures across tools. Since, the notion of “quality” with respect to SKOS vocabularies naturally varies across application domains it is hard, if not impossible, to come up with a general definition of quality. However, we believe that vocabulary designers can apply a set of \textbf{formal quality criteria} on given vocabularies in order to assess whether or not a vocabulary fulfills their requirements. Since these criteria are based on a formal definition, they can be calculated by automated tools, which helps vocabulary designers in assessing the quality of available or newly created vocabularies and provide important feedback to the overall vocabulary design process.

The main contributions of this work can be summarized as follows:
\begin{itemize}
	\item We derived a \textbf{quality criteria catalogue} from existing literature, tools, and public mailing list discussions in the community. We \textbf{formalized} each criterion and implemented them in the \textbf{qSKOS quality assessment tool}.

	\item We \textbf{analyzed} a representative \textbf{set of existing SKOS vocabularies} to learn about possible quality issues in existing Web vocabularies.
\end{itemize}

The results from our studies show that... TBD

\section{A Quality Model for SKOS}
SKOS is a language for defining vocabularies in the Web of Data and therefore based on the Open World Assumption. Established quality notions from closed-world systems, such as constraints, referential integrity or schema validation, don’t hold anymore, because the available information may be incomplete and facts that are not explicitly stated cannot be determined as true or false. While trust and provenance models for Web data are being developed \cite{Omitola2011,Hartig2009}, content-based and hand-crafted heuristics are currently used to evaluate quality in Linked Data sets \cite{Heath2011}.

Based on the findings from a literature survey we now define a set of quantifiable quality criteria for SKOS vocabularies. For each \textbf{criterion} we explain our design rationale, and provide a semi-formal definition, which is based on a set-theoretic conception of the SKOS model. Tools can implement these criteria as functions defined on-top of a the baseline SKOS model.

\begin{definition} \textbf{The SKOS Baseline Model} is defined as follows:
Let R be the set of all resources, as defined in [Web Architecture Document] and \[V = (C, A, LL, LR, SR, AR)\] be one specific SKOS vocabulary, with

\begin{itemize}
	\item \(C \subseteq R\) being the set of \textbf{conceptual resources} of type \texttt{skos:Concept}.
	\item \(A \subseteq R\) being the set of aggregation and grouping resources in SKOS, hence \texttt{skos:ConceptSchema} and \texttt{skos:Collection}.
	\item LL being the set \textbf{lexical labels}, which are instances of RDF plain literals.
	\item \(LR \subseteq C \times LL\) being the set of \textbf{lexical relations} associating conceptual resources with lexical labels, hence instances of \texttt{skos:prefLabel}, \texttt{skos:altLabel}, or \texttt{skos:hiddenLabel}.
\item \(SR \subseteq C \times C\) being the set of \textbf{semantic relations} associating conceptual resources with concepts, hence instances of \texttt{skos:semanticRelation} and subproperties thereof.
\item \(AR \subseteq C \times A\) being the set of \textbf{aggregation (and grouping) relations} associating conceptual resources with instances of aggregation and grouping resources.
\end{itemize}
\end{definition}

\subsection{Graph-based Quality Criteria / Network Measures}
The following quality criteria are all derived from a graph-based view on a given vocabulary \(v\), which we can then consider as a directed graph \(G=(N,E)\) consisting of a set of nodes \(N\) and edges \(E\).

\subsubsection{Loose Concepts} are very similar to what is also referred to as ``orphan terms” in the literature, which are terms without any associative or hierarchical relationships \cite{Z39.19:2005,Hedden2010}. In a SKOS vocabulary, we define a loose concept as a concept that has no semantic or aggregation relations with other conceptual, aggregation or grouping resources. Although they might have attached lexical labels, they lack valuable context information. Checking for orphan terms is common in existing thesaurus development applications and suggested by \cite{Z39.19:2005}. Depending on the vocabulary development policy, loose concepts may be tolerated or considered to be a defect.

\begin{definition}
To calculate the number of loose concepts, we let \(G = (C, SR \cup AR)\)  and apply \(deg(c)\) as a function that calculates the degree, hence the number of in- and outgoing edges of a node n in a graph g. We can then define this quality criterion as a function \(looseConcepts : G \rightarrow \mathbb{N}_{0}\), with \[looseConcepts(g) = \left\|\left\{c \in C : deg(c) = 0\right\}\right\|\]
\end{definition}

\subsubsection{Weakly Connected Components} cause the vocabulary to split into separate disconnected components. This is very similar to the ``loose concepts” criterion introduced before and might be caused by incomplete data acquisition, ``forgotten" test data, outdated terms, accidental deletion of relations and the like. In a practical setting, existence of weakly connected components could render the vocabulary less suitable for operations that rely on navigating a strongly connected vocabulary structure, such as query expansion or suggestion of related terms.

\begin{definition}
To evaluate this criterion we consider a graph \(G = (AC, SRAR)\). The function \(components(g) : G \rightarrow  \mathfrak{P}(G)\) calculates all weakly connected components of a graph by replacing all directed edges by undirected edges and utilizing “Tarjan’s algorithm” \cite{Hopcroft1973}. Based on that function, we can express this criterion as a function \(wcc : G \rightarrow \mathbb{N}_{0}\), with \[wcc(g) = \left\|components(g)\right\|\]
\end{definition}

\subsubsection{Cyclic Hierarchical Relations} have been mentioned numerous times in thesaurus development literature. \cite{Soergel2002} suggests a ``check for hierarchy cycles" since they ``throw the program for a loop in the generation of a complete hierarchical structure”. However, the SKOS documentation does not state, how exactly hierarchical relationships are to be interpreted. There exist common forms like, e.g., “generic-specific”, “instance-of” or “whole-part” \cite{Hedden2010,Harpring2010,Aitchison2000}. Hierarchical cycles for some of these interpretations in various domains or usage scenarios might be perfectly valid, whereas they would be treated as failures in others.

\begin{definition}
In the context of this work, we define a cycle in a SKOS vocabulary as a logical contradiction in the broader/narrower hierarchy. Let \(HB \subseteq SR\) be the set of broader hierarchical relations associating two conceptual resources with instances of \texttt{skos:broader}, \texttt{skos:broaderTransitive}, \texttt{skos:broadMatch} or subproperties thereof. \(HB\) also associates conceptual resources related with instances \texttt{skos:narrower}, \texttt{skos:narrowTransitive}, \texttt{skos:narrowMatch} or subproperties thereof in reverse order. Likewise, \(HN \subseteq SR\) can be defined as the set of narrower hierarchical relations. We consider two graphs \(GB = (C,HB)\) and \(GN=(C,HN)\) and a function \(cycleNodes:GB \cup GN \rightarrow \mathfrak{P}(C)\) identifying all nodes in a graph, that are part of a cycle. Furthermore we consider a function \(stronglyConnectedSets: GB \cup GN \rightarrow \mathfrak{P}(C)\) that calculates all strongly connected undirected subgraphs of a graph. This criterion can be defined as a function \(chr:GB \cup GN \rightarrow \mathbb{N}_{0}\) with \[chr(g)=\left\|\left\{s \in stronglyConnectedSets(g) : cn \in cycleNodes(g), cn \in s\right\}\right\|\]
\end{definition}

\subsection{Structure-Centric Quality Criteria / Structural Measures}

\subsubsection{Redundant Associative Relations} \cite{ISO25964-1:2011} suggests that terms that share a common broader term but don’t have an overlapping meaning, should not be related associatively. This is also advocated by \cite{Hedden2010} and \cite{Aitchison2000} who also mentions ``the risk that thesaurus compilers may overload the thesaurus with valueless relationships", having a negative effect on precision.

\begin{definition}
Let \(ASR \subseteq SR\) be the set of associative relations in SKOS, namely \texttt{skos:related} and \texttt{skos:relatedMatch}. These relations are symmetric, so \((c1,c2) \in ASR \Rightarrow (c2,c1) \in ASR\). Formally, this criterion can be expressed as a function \(rar:V \rightarrow \mathbb{N}_{0}\) with \[rar(v)=\left\|\left\{(c1,c2) \in ASR : c3 \in C, (c1,c3) \in HB, (c2,c3) \in HB\right\}\right\|\]
\end{definition}

\subsubsection{Number of Hierarchically and Associatively Related Concepts} \cite{Aitchison2000} mentions a ``common error in manually produced thesauri" when concepts are related both hierarchically and associatively (F1.3.2). \cite{Hedden2010} also proposes to check for ``pairs of terms that have more than one kind of relationship" as a feasible rule, enforced by thesaurus management software.

\begin{definition}
This criterion can be expressed as a function \(har:V \righarrow \mathbb{N}_{0}\) with \[har(v)=\left\|\left\{c \in C : c' \in C, (c,c') \in HB \cup HN, (c,c') \in ASR\right\}\right\|\]
\end{definition}

\subsubsection{Unidirectionally Related Concepts} \cite{Z39.19:2005} suggests thesaurus maintenance software to automatically add reciprocal relations (e.g., broader/narrower, related) to the resulting thesaurus. Vocabularies created in that manner are expected, e.g., to be better browsable for humans and provide better search results when used in systems without reasoning support.

\begin{definition}
Let \(ISR \subseteq R \times R\) be a set that associates resources defined in \(V\) that are related by a SKOS property that has an inverse SKOS property defined. Furthermore, ISR also associates resources that are related by a symmetric SKOS property. In an ideal setting \((r,r') \in ISR \iff (r',r) \in ISR\). Thus we can express this criterion as function \(urc:V \rightarrow \mathbb{N}_{0}\) with \[urc(v)=\left\|\left\{c \in C : r \in R, (c,r) \in ISR, (r,c) \notin ISR\right\}\right\|\]
\end{definition}

\subsubsection{Solely Transitively Related Concepts} Two concepts are related using only transitive hierarchical relations which are, according to the SKOS reference document, ``not used to make assertions". Transitive hierarchical relations in SKOS are meant to be infered by the vocabulary user, which is reflected in the SKOS schema by, e.g., broader being a subproperty of broaderTransitive.

\begin{definition}
Let \(TR \subseteq SR\) be the set of all conceptual resources being transitively related by \texttt{skos:broaderTransitive} respectively \texttt{skos:narrowerTransitive}, not by subproperties thereof. Furthermore, let \(HR \subseteq SR\) be the set of all conceptual resources being related by \texttt{skos:broader} and \texttt{skos:narrower} with \((c,c') \in HR \Rightarrow (c',c) \in HR\). We can write this criterion as a function \(str:V \rightarrow \mathbb{N}_{0}\) with \[str(v)=\left\|\left\{c \in C : c' \in C, (c,c') \in TR, (c,c') \notin HR\right\}\right\|\]
\end{definition}

\subsection{Linked Data Specific Criteria / Linked Data Measures}
SKOS has been developed for creating knowledge organization systems that are used ``within the framework of the semantic web" \cite{Bernerslee2001}. Therefore we identified some criteria targeting issues related to best practises in publishing Linked Data, the underlying technology for establishing a Semantic Web \cite{Heath2011}.

\subsubsection{Concept External Link Average} The ``Web of Data" consists of globally interconnected datasets, ``enabling seamless connections between data sets" \cite{Heath2011}. These links can be established between various SKOS vocabularies by ``external links", i.e., references from conceptual resources in a ``source vocabulary" to resources identified by an HTTP URI and available at a different host.

\begin{definition}
Let \(EL=C \times R\) be the set of external links with \(host(c) \neq host(r)\). This criterion can then be written as a function \(elc:V \righarrow \mathbb{ℚ}\) with \[elc(v)=\frac{\left\|EL\right\|}{\left\|C\right\|}\]
\end{definition}

\subsection{HTTP URI Scheme Violation} One of the key concept of Linked Data is that URIs should be dereferencable in order to browse through data in the same way as through ``traditional" Web pages. Using HTTP URIs therefore is the logical consequence and mentioned numerous time in literature \cite{Heath2011}. However, some vocabularies use other URI schemes such as URNs and DOIs when defining resources, which impedes data lookup and crawling purposes.

\begin{definition}
Let \(VR \subseteq C \cup A\) be the set of vocabulary resources identified by an URI (not by a blank node). We can define a function \(httpScheme:R \rightarrow \left\{0,1\right\}\) with \(httpScheme(r)=1\) if the scheme part of the URI is http or https, 0 otherwise. Then this criterion can be written as a function \(usv:V \rightarrow \mathbb{N}_{0}\) with \[usv(v)=\left\|\left\{vr \in VR : httpScheme(vr)=0\right\}\right\|\]
\end{definition}

\subsubsection{Link Target Unavailability} Just as in the ``traditional" Web, resources in the Web of Data might also be unavailable, i.e. resolving their URI leads to an erroneous HTTP response or no response at all. An erroneous HTTP response in that case can be defined as a response code (after a possible redirection) other than 200. Since these ``broken links" hinder information gathering, they should be avoided.

\begin{definition}
Let \(LR \subseteq R\) be the set of all resources being related to conceptual, aggregational and grouping resources. The function \(resp:R \rightarrow \mathbb{N}\) assigns a resource its HTTP response code after following all redirections. We can therefore write this criterion as function \(ltu:V \rightarrow \mathbb{N}_{0}\) with \[ltu(v)=\left\|\left\{r \in LR \cup VR : resp(r)=200\right\}\right\|\]
\end{definition}

\subsubsection{Average Concept In-degree} A periodically updated version of a diagram visualizing ``the data sets in the LOD cloud as well as their interlinkage relationships” is published online \footnote{http://www4.wiwiss.fu-berlin.de/lodcloud/state/}. In these diagrams, DBpedia \footnote{http://dbpedia.org/About} is shown as the most intensely linked resource, i.e. a large number of vocabularies link to resources in that dataset. Therefore a dataset with a large number of other datasets referencing it, can be assumed to be of great value for the community. Accordingly, estimating the number of ``incoming links" of SKOS concepts defined in a vocabulary (henceforth called the in-degree of a concept) would give some hint about how established or accepted a controlled vocabulary is. A method to carry out these estimations would be utilizing existing Linked Data indices like, e.g., Sindice \footnote{http://sindice.com/} that provide a SPARQL endpoint.

\begin{definition}
Let \(U\) be the universe of all datasets on the Web and \(V \in U\). The set \(RR \subseteq R \times R\) contains all resources in distinct datasets related by an RDF property with \((r1,r2) \in RR \Rightarrow host(r1) \neq host(r2)\). This criterion can be written as a function \(aci:V \rightarrow \mathbb{ℚ}\) with \[aci(v)=\frac{\left\|\left\{r \in R : c \in C, (r,c) \in RR\right\}\right|}{\left\|C\right\|}\] 
\end{definition}

\subsection{SKOS-Specific Criteria / SKOS Semantic Measures}

\subsection{Labeling Issues}

\subsection{Other Criteria}

\section{Analysis of Existing SKOS Vocabularies}

\subsection{The qSKOS Quality Checking Tool}

\subsection{Vocabularies}

\subsection{Results}

\section{Related Work}

\section{Conclusions and Future Work}

\bibliography{lit}
\bibliographystyle{splncs03}

\end{document}

%!TEX root = main.tex

\section{Quality Issues in SKOS Vocabularies}\label{sec:criteria}

% Briefly describe methodology

We identified an initial set of possible quality issues in SKOS vocabularies by reviewing literature and manually examining existing vocabularies. We strictly focused on issues that can be formally expressed and thus automatically measured. However, common graph measures like, e.g. hierarchy depth or node centrality have been omitted due to lack of evidence on their general influence on vocabulary quality.
We published our findings in the qSKOS wiki and requested feedback from experts via public mailing lists and informal face to face discussions. Based on the received responses, we then selected a subset of these issues and translated them into computable quality checking functions. Each function takes a given SKOS vocabulary and an optional vocabulary namespace as input and finds all resources that match the corresponding quality issue. For the purpose of this work, we define a SKOS vocabulary as follows:

\spnewtheorem*{mydef}{Definition}{\bfseries}{\itshape}
\begin{mydef}[SKOS Vocabulary]
Let a SKOS vocabulary be a tuple of the form $V = \langle IR, C, AC, SR, LV, CS \rangle$, with \(IR = I_{CEXT}(\texttt{rdfs:Resource}^\mathcal{I})\) being the set of \textbf{resources}, \(C \subseteq IR\) with \(C = I_{CEXT}(\texttt{skos:Concept}^\mathcal{I})\) being the set of \textbf{concepts}, \(AC \subseteq C\) being the set of \textbf{authoritative concepts}, which are all concepts that are identified by URIs in the vocabulary namespace, \(SR = I_{EXT}(\texttt{skos:semanticRelation}^\mathcal{I})\) being the set of \textbf{semantic relations} associating concepts with one another, $LV \subseteq I_{CEXT}(\texttt{rdfs:Literal}^\mathcal{I})$ being the set of untyped \textbf{plain literals}, and \(CS = I_{CEXT}(\texttt{skos:ConceptScheme}^\mathcal{I})\) being the set of \textbf{concept schemes}. Further, we let $V$ be the fully entailed RDFS interpretation of the underlying RDF graph. We enrich $V$ by entailment of \texttt{owl:inverseOf} properties as well as instances of \texttt{owl:TransitiveProperty} and \texttt{owl:SymmetricProperty} defined by the formal OWL semantics of SKOS~\cite{SkosReference2008}.
\end{mydef}

In the following, we explain the origins and design rationale for each quality issue and explain how the corresponding quality checking function works. For better readability and due to lack of space we provide only semi-formal definitions and refer to the source code of the qSKOS tool for further details.

% where to put the example for authoritative concepts?
%For example, \texttt{PXV} links \url{http://www.peroxisomekb.nl/v1.6/pxv/C000297} to \url{http://www.bioexpert.nl/mesh/2010/concept/M0028264} using \texttt{skos:closeMatch}. Both are \texttt{skos:Concept}s, but the latter is no authoritative concept because its URI does not belong to the \texttt{PXV} namespace.

% LABELING AND DOCUMENTATION ISSUES

\subsection{Labeling and Documentation Issues}

\paragraph{Omitted or Invalid Language Tags}

% Origin and rationale
SKOS defines a set of properties that link resources with RDF Literals, which are plain text natural language strings with an optional language tag. This includes the labeling properties {\texttt{rdfs:label}, \texttt{skos:prefLabel}, \texttt{skos:altLabel}, \texttt{skos:hiddenLabel} and also SKOS documentation properties, such as \texttt{skos:note} and subproperties thereof. Literals should be tagged consistently~\cite{Vrandecic2010}, because omitting language tags or using non-standardized, private language tags in a SKOS vocabulary could unintentionally limit the result set of language-dependent queries.
% Function defintion
A SKOS vocabulary can be checked for omitted and invalid language tags by iterating over all resources in $IR$ and finding those that have labeling or documentation property relations to plain literals in $LV$ with missing or invalid language tags, i.e., tags that are not defined in RFC3066\footnote{Tags for the Identification of Languages \url{http://www.ietf.org/rfc/rfc3066.txt}}.

\paragraph{Incomplete Language Coverage}

% Origin and rationale
The set of language tags used by the literal values linked with a concept should be the same for all concepts. If this is not the case, appropriate actions like, e.g., splitting concepts or introducing scope notes should be taken by the creators. This is particularly important for applications that rely on internationalization and translation use cases.
% Function defintion
Affected concepts can be identified by first extracting the global set of language tags used in a vocabulary from all literal values in $LV$, which are attached to a concept in $C$. In a second iteration over all concepts, those having a set of language tags that is not equal to the global language tag set are returned.

\paragraph{Undocumented Concepts}

% Origin and rationale
Svenonius~\cite{Svenonius1997} advocates the ``inclusion of as much definition material as possible'' and the SKOS Reference~\cite{SkosReference2008} defines a set of ``documentation properties'' intended to hold this kind of information.
% Function defintion
To identify all undocumented concepts, we iterate over all concepts in $C$ and collect those that do not use any of these documentation properties.

\paragraph{Label Conflicts}

% Origin and rationale
The SKOS Primer~\cite{Isaac2009} recommends that ``no two concepts have the same preferred lexical label in a given language when they belong to the same concept scheme''. This issue could affect application scenarios such as auto-completion, which proposes labels based on user input. Although these extra cases are acceptable for some thesauri, we generalize the above recommendation and search for all concept pairs with their respective \texttt{skos:prefLabel}, \texttt{skos:altLabel} or \texttt{skos:hiddenLabel} property values meeting a certain similarity threshold defined by a function $sim:LV \times LV \rightarrow [0,1]$.
% Function defintion
The default, built-in similarity function checks for case-insensitive string equality with a threshold equal to 1. Label conflicts can be found by iterating over all (authoritative) concept pairs $AC \times AC$, applying $sim$ to every possible label combination, and collecting those pairs with at least one label combination meeting or exceeding a specified similarity threshold. We handle this issue under the Closed World Assumption, because data on concept scheme membership may lack and concepts may be linked to concepts with similar labels in other vocabularies.
% Since data on concept scheme membership may lack because of the Open World Assumption, we handle this issue under the Closed World Assumption.

% STRUCTURAL METRICS

\subsection{Structural Issues}

\paragraph{Orphan Concepts}

% Origin and rationale
are motivated by the notion of ``orphan terms'' in the literature~\cite{Hedden2010}, i.e., terms without any associative or hierarchical relationships. Checking for such terms is common in thesaurus development and also suggested by~\cite{Z39.19:2005}. Since SKOS is concept-centric, we understand an orphan concept as being a concept that has no semantic relation $sr \in SR$ with any other concept. Although it might have attached lexical labels, it lacks valuable context information, which can be essential for retrieval tasks such as search query expansion.
% Function defintion
Orphan concepts in a SKOS vocabulary can be found by iterating over all elements in $C$ and selecting those without any semantic relation to another concept in $C$.

\paragraph{Weakly Connected Components}

% Origin and rationale
A vocabulary can be split into separate ``clusters'' because of incomplete data acquisition, deprecated terms, accidental deletion of relations, etc. This can affect operations that rely on navigating a connected vocabulary structure, such as query expansion or suggestion of related terms.
% Function defintion
Weakly connected components are identified by first creating an undirected graph that includes all non-orphan concepts (as defined above) as nodes and all semantic relations $SR$ as edges. ``Tarjan's algorithm''~\cite{Hopcroft1973} can then be applied to find all connected components, i.e., all sets of concepts that are connected together by (chains of) semantic relations.

\paragraph{Cyclic Hierarchical Relations}

% Origin and rationale
is motivated by Soergel et al.~\cite{Soergel2002} who suggest a ``check for hierarchy cycles'' since they ``throw the program for a loop in the generation of a complete hierarchical structure''. Also Hedden~\cite{Hedden2010}, Harpring~\cite{Harpring2010} and Aitchison et al.~\cite{Aitchison2000} argue that there exist common forms like, e.g., ``generic-specific'', ``instance-of'' or ``whole-part'' where cycles would be considered a logical contradiction.
% Function defintion
Cyclic relations can be found by constructing a graph with the set of nodes being $C$ and the set of edges being all \texttt{skos:broader} relations.

\paragraph{Valueless Associative Relations}

% Origin and rationale
The ISO/DIS 25964-1 standard \cite{ISO25964-1:2011} suggests that terms that share a common broader term should not be related associatively if this relation is only justified by the fact that they are siblings. This is advocated by Hedden~\cite{Hedden2010} and Aitchison et al.~\cite{Aitchison2000} who point out ``the risk that thesaurus compilers may overload the thesaurus with valueless relationships'', having a negative effect on precision.
% Function defintion
This issue can be checked by identifying concept pairs $C \times C$ that share the same broader or narrower concept while also being associatively related by the property \texttt{skos:related}.

\paragraph{Solely Transitively Related Concepts}

% Origin and rationale
Two concepts that are explicitly related by \texttt{skos:broaderTransitive} and/or \texttt{skos:narrowerTransitive} can be regarded a quality issue because, according to \cite{SkosReference2008}, these properties are ``not used to make assertions''. Transitive hierarchical relations in SKOS are meant to be inferred by the vocabulary consumer, which is reflected in the SKOS ontology by, for instance, \texttt{skos:broader} being a subproperty of \texttt{skos:broaderTransitive}. 
% Function definition
This issue can be detected by finding all concept pairs $C \times C$ that are directly related by \texttt{skos:broaderTransitive} and/or \texttt{skos:narrowerTransitive} properties but not by (chains of) \texttt{skos:broader} and \texttt{skos:narrower} subproperties.

\paragraph{Omitted Top Concepts}

% Origin and rationale
The SKOS model provides concept schemes, which are a facility for grouping related concepts. This helps to provide ``efficient access''~\cite{Isaac2009} and simplifies orientation in the vocabulary. In order to provide entry points to such a group of concepts, one or more concepts can be marked as top concepts.  
% Function definition
Omitted top concepts can be detected by iterating over all concept schemes in $CS$ and collecting those that do not occur in relations established by the properties \texttt{skos:hasTopConcept} or \texttt{skos:topConceptOf}.

\paragraph{Top Concept Having Broader Concepts}

% Origin and rationale
Allemang et al. \cite{Allemang2011} propose to ``not indicate any concepts internal to the tree as top concepts'', which means that top concepts should not have broader concepts. 
% Function definition
Affected resources are found by collecting all top concepts that are related to a resource via a \texttt{skos:broader} statement and not via \texttt{skos:broadMatch}---mappings are not part of a vocabulary's ``intrinsic'' definition and a top concept in one vocabulary may perfectly have a broader concept in another vocabulary.

% LINKED-DATA SPECIFIC ISSUES

\subsection{Linked Data Specific Issues}\label{subsec:ld_issues}

\paragraph{Missing In-Links}

% Origin and rationale
When vocabularies are published on the Web, SKOS concepts become linkable resources. Estimating the number of in-links and identifying the concepts without any in-links, can indicate the importance of a concept.
% Function definition
We estimate the number of in-links by iterating over all elements in $AC$ and querying the Sindice\footnote{\url{http://sindice.com/} indexes the Web of Data, which is composed of pages with semantic markup in RDF, RDFa, Microformats, or Microdata. Currently it covers approximately 230M documents with over 11 billion triples.} SPARQL endpoint for triples containing the concept's URI in the object part. Empty query results are indicators for missing in-links.

\paragraph{Missing Out-Links}

% Origin and rationale
In the Linked Data context, SKOS concepts should also be linked with other related concepts on the Web, ``enabling seamless connections between data sets''\cite{Heath2011}. Similar to \emph{Missing In-Links}, this issue identifies the set of all authoritative concepts that have no out-links.
% Function definition
It can be computed by iterating over all elements in $AC$ and returning those that are not linked with any non-authoritative resource.

\paragraph{Broken Links}

% Origin and rationale
As we discussed in detail in our earlier work~\cite{Popitsch:2010:DHB:1772690.1772768}, this issue is caused by vocabulary resources that return HTTP error responses or no response when being dereferenced. An erroneous HTTP response in that case can be defined as a response code other than 200 after possible redirections. Just as in the ``document'' Web, these ``broken links'' hinder navigability also in the Linked Data Web and and should therefore be avoided. 
% Function definition
Broken links are detected by iterating over all resources in $IR$, dereferencing their HTTP URIs, following possible redirects, and including unavailable resources in the result set.

\paragraph{Undefined SKOS Resources}

% Origin and rationale
The SKOS model is defined within the namespace \url{http://www.w3.org/2004/02/skos/core#}. However, some vocabularies use resources from within this namespace, which are unresolvable for two main reasons: vocabulary creators ``invented'' new terms within the SKOS namespace instead of introducing them in a separate namespace, or they use ``deprecated'' SKOS elements like \texttt{skos:subject}.
% Function definition
Undefined SKOS resources can be identified by iterating over all resources in $IR$ and returning those (i) that are contained in the list of deprecated resources\footnote{See \url{http://www.w3.org/TR/skos-reference/#namespace}} or (ii) are identified by a URI in the SKOS namespace but are not defined in the current version of the SKOS ontology.

%!TEX root = main.tex

\section{Quality Issues in SKOS Vocabularies}\label{sec:criteria}

% Briefly describe methodology

We identified an initial set of possible quality issues in SKOS vocabularies by reviewing literature and manually examining existing vocabularies. We published our findings online\footnote{\url{https://github.com/cmader/qSKOS/wiki/Quality-Criteria-for-SKOS-Vocabularies}} and requested feedback from experts via public mailing lists and informal face to face discussions. Based on the received responses, we then selected a subset of these issues and formalized them into computable \textbf{quality checking functions}. Each function operates on a given SKOS vocabulary, which we define, for the purpose of this work, as follows:

\paragraph{\textbf{Definition (SKOS Vocabulary).}}
Let a SKOS vocabulary be a tuple of the form $V = \langle IR, C, SR, LV, CS \rangle$, with \(IR = I_{CEXT}(\texttt{rdfs:Resource}^\mathcal{I})\) being the set of \textbf{resources}, \(C \subseteq IR\) with \(C = I_{CEXT}(\texttt{skos:Concept}^\mathcal{I})\) \todo{CM}{check exponent}being the set of \textbf{concepts}, \(SR = I_{EXT}(\texttt{skos:semanticRelation}^\mathcal{I})\) being the set of \textbf{semantic relations} associating concepts with one another, $LV \subseteq I_{CEXT}(\texttt{rdfs:Literal}^\mathcal{I})$ being the set of untyped \textbf{plain literals}, and \(CS I_{CEXT}(\texttt{skos:ConceptScheme}^\mathcal{I})\) being the set of \textbf{concept schemes}. Further, we let $V$ be the fully entailed RDFS interpretation of the underlying RDF graph, enriched by entailment of \texttt{owl:inverseOf} properties as well as subproperties of \texttt{owl:TransitiveProperty} and \texttt{owl:SymmetricProperty}. 

A checking function is applied on a given SKOS vocabulary and finds all vocabulary resources that match the corresponding quality issue. We can map each issue to an abstract function $f:V \rightarrow \mathcal{P}(IR)$ that relates a given SKOS vocabulary to a subset of its resources. Each quality issue we found can be implemented as a realization of that abstract function and possibly take additional parameters. Statistical analysis can be computed on top of the result set.

In the following, we explain the origins and design rationale for each quality issue and define the corresponding quality checking function. For better readability and due to lack of space we provide only semi-formal definitions and refer to the \sout{online documentation and the} implementation source code\footnote{\url{https://github.com/cmader/qSKOS/}} for further details.

% LABELING AND DOCUMENTATION ISSUES

\subsection{Labeling and Documentation Issues}

\paragraph{Omitted or Invalid Language Tags}

% Origin and rationale
SKOS defines a set of properties that link resources with RDF Literals, which are plain text natural language strings with an optional language tag. This includes the labeling properties \texttt{skos:prefLabel}, \texttt{skos:altLabel}, \texttt{skos:hiddenLabel} and also SKOS documentation properties, such as \texttt{skos:note} and subproperties thereof. Literals should be tagged consistently~\cite{Vrandecic2010}, because omitting language tags or using non-standardized, private language tags in a SKOS vocabulary could unintentionally limit the result set of language-dependent queries.
% Function defintion
A SKOS vocabulary can be checked for omitted and invalid language tags by iterating over all resources in $IR$ and finding those that have relations to plain literals (using the above mentions SKOS properties) with missing or invalid language tags, hence tags that are not defined in RFC3066\footnote{RFC 3066: Tags for the Identification of Languages \url{http://www.ietf.org/rfc/rfc3066.txt}}.

\paragraph{Incomplete Language Coverage}

% Origin and rationale
Soergel et al.~\cite{Soergel1997} point out the problem that ``...a concept lexicalized in one language may not be lexicalized in another and that the terms do exist often vary slightly in meaning, possibly giving rise to different relationships.'' For such cases, the authors suggest to revise the vocabulary structure and to ``clearly instruct indexers in all languages how the term is to be used''. In the context of SKOS, this means that the set of language tags used by the literal values linked with a concept should be checked to be the same for all concepts. If this is not true for a vocabulary, appropriate actions like, e.g., splitting concepts or introducing scope notes should be taken by the creators. This is particularly important for applications that rely on internationalization and translation use cases.
% Function defintion
Affected concepts can be identified by first extracting the global set of language tags used in a vocabulary from all literal values attached to all concepts in $C$. In a second iteration over all concepts, all concepts having a set of language tags that is not equal to the global language set are returned.

\paragraph{Undocumented Concepts}

% Origin and rationale
Svenonius~\cite{Svenonius1997} advocates the ``inclusion of as much definition material as possible'' and the SKOS Reference~\cite{SkosReference2008} defines a set of properties, denoted as ``documentation properties'', that are intended to hold this kind of information.
% Function defintion
To identify all undocumented concepts, we iterate over all concepts in $C$ and collect those that do not use any of these documentation properties.

\paragraph{Label Conflicts}

% Origin and rationale
The SKOS Primer~\cite{Isaac2009} recommends that ``no two concepts have the same preferred lexical label in a given language when they belong to the same concept scheme''. We reviewed existing vocabularies and applications and found that (i) many vocabularies don't use concept schemes and (ii) identical \texttt{skos:altLabels} or \texttt{skos:hiddenLabels} can also negatively affect use cases such as auto-completion, which proposes labels based on the user input.
% Function defintion
Thus, we generalize this recommendation and search for all concept pairs with their respective \texttt{skos:prefLabel}, \texttt{skos:altLabel} or \texttt{skos:hiddenLabel} property values meeting a certain similarity threshold defined by a function $sim:LV \times LV \rightarrow [0,1]$. The default, built-in similarity function checks for case-insensitive string equality with a threshold equal to 1. The function iterates over all concept pairs $C \times C$, applies $sim$ to every possible label combination, and collects those pairs with at least one label combination that meets or exceeds a specified similarity threshold.

% STRUCTURAL METRICS

\subsection{Structural Issues}

\paragraph{Orphan Concepts}

% Origin and rationale
This issue is motivated by the notion of ``orphan terms'' in the literature (cf.,~\cite{Hedden2010}), which are are terms without any associative or hierarchical relationships. Checking for such terms is common in existing thesaurus development applications and also suggested by~\cite{Z39.19:2005}. Since SKOS is concept-centric, we understand an orphan concept as being a concept that has no semantic relation $sr \in SR$ with any other concept. Although it might have attached lexical labels, it lacks valuable context information, which can be essential for retrieval tasks such as search query expansion.
% Function defintion
Orphan concepts in a SKOS vocabulary can be found by iterating over all elements in $C$ and selecting those without any semantic relation to another conceptual resource in $C$.

\paragraph{Weakly Connected Components}

% Origin and rationale
A vocabulary can be split into separate ``clusters'', which can be caused by incomplete data acquisition, ``forgotten'' test data, outdated terms, accidental deletion of relations, etc. This can affect operations that rely on navigating a connected vocabulary structure, such as query expansion or suggestion of related terms.
% Function defintion
Weakly connected components are identified by first creating an undirected graph that includes all non-orphan concepts (as defined above) as nodes and all semantic relations $SR$ as edges. ``Tarjan’s algorithm''~\cite{Hopcroft1973} can then be applied to find all connected components, hence all sets of concepts that are connected by semantic relations.

\paragraph{Cyclic Hierarchical Relations}

% Origin and rationale
is motivated by Soergel et al.~\cite{Soergel2002} who suggest a ``check for hierarchy cycles'' since they ``throw the program for a loop in the generation of a complete hierarchical structure''. Also Hedden~\cite{Hedden2010}, Harpring~\cite{Harpring2010} and Aitchison et al.~\cite{Aitchison2000} argue that there exist common forms like, e.g., ``generic-specific'', ``instance-of'' or ``whole-part'' where cycles would be considered a logical contradiction.
% Function defintion
Cyclic relations can be found by constructing a graph with the set of nodes being $C$ and the set of edges being all \texttt{skos:broader} relations between these concepts.

\paragraph{Valueless Associative Relations}

% Origin and rationale
The ISO/DIS 25964-1 standard \cite{ISO25964-1:2011} suggests that terms that share a common broader term but do not have an overlapping meaning, should not be related associatively. This is advocated by Hedden~\cite{Hedden2010} and Aitchison et al.~\cite{Aitchison2000} who point out ``the risk that thesaurus compilers may overload the thesaurus with valueless relationships'', having a negative effect on precision.
% Function defintion
This issue can be checked by identifying the set of pairs of conceptual resources that share the same broader or narrower concept while also being associatively related by the property \texttt{skos:related}.

\paragraph{Solely Transitively Related Concepts}

% Origin and rationale
Two concepts that are explicitly related by \texttt{skos:broaderTransitive} and/or \texttt{skos:narrowerTransitive} can be regarded a quality issue because, according to \cite{SkosReference2008}, these properties are ``not used to make assertions''. Transitive hierarchical relations in SKOS are meant to be inferred by the vocabulary consumer, which is reflected in the SKOS schema by, for instance, \texttt{skos:broader} being a subproperty of \texttt{skos:broaderTransitive}. 
% Function definition
This issue can be detected by iterating over all concept pairs $C \times C$ that are directly related by \texttt{skos:broaderTransitive} and/or \texttt{skos:narrowerTransitive} properties but none of their subproperties.

\paragraph{Omitted Top Concepts}

% Origin and rationale
The SKOS model provides concept schemes, which are a facility for grouping related concepts. This helps to provide ``efficient access''~\cite{Isaac2009} and simplifies orientation in the vocabulary. In order to provide entry points to such a group of concepts, one or more concepts can be marked as top concepts.  
% Function definition
Omitted top concepts can be detected by iterating over all concept schemes in $CS$ and collecting those that do not occur in relations established by the properties \texttt{skos:hasTopConcept} or \texttt{skos:topConceptOf}.

\paragraph{Top Concept Having Broader Concepts}

% Origin and rationale
Allemang et al. \cite{Allemang2011} propose to ``not indicate any concepts internal to the tree as top concepts'', which means that top concepts should not have broader concepts. 
% Function definition
Affected resources can be found by iterating over all \texttt{skos:TopConcept}s and collecting those that are related to a resource via a \texttt{skos:broader} property.

% LINKED-DATA SPECIFIC ISSUES

\subsection{Linked Data Specific Issues}\label{subsec:ld_issues}

\paragraph{Missing In-Links}

% Origin and rationale
SKOS concepts become linkable resources, when vocabularies are being published on the Web. Estimating the number of in-links and identifying the concepts without any in-links, can be an important indicator for the importance of a concept.
% Function definition
To estimate the number of in-links we only consider the set $AC \subseteq C$, which consists of all ``authoritative concepts'', hence concepts that have an URI in the vocabulary namespace. The number of in-links referencing an authoritative concept can be estimated by iterating over all elements in $AC$ and querying the Sindice\footnote{\url{http://sindice.com/}} SPARQL endpoint for triples containing the concept's URI in the object part. Empty query results are indicators for missing in-links and are thus included in the result set of this function.

\paragraph{Missing Out-Links}

% Origin and rationale
In the Linked Data context, SKOS concepts should be linked with other related concepts on the Web, ``enabling seamless connections between data sets''\cite{Heath2011}. Similar to \emph{Missing In-Links}, this issue identifies the set of all concepts that have no out-links.
% Function definition
It can be computed by iterating over all elements $c \in C$ and checking the host part of all related resource URIs. If none of these host names differ from the host part of $c$'s URI, it has no out-links and is thus included in the resources identified by this issue.

\paragraph{Broken Links}

% Origin and rationale
As we discussed in detail in our earlier work~\cite{Popitsch:2010:DHB:1772690.1772768}, this issue is caused by vocabulary resources that return HTTP error responses or no response when being dereferenced. An erroneous HTTP response in that case can be defined as a response code other than 200 when taking possible redirections into account. Just as in the ``document'' Web, these ``broken links'' hinder navigability also in the Linked Data Web and and should therefore be avoided. 
% Function definition
Broken links are detected by iterating over all resources in $IR$, dereferencing their HTTP URIs, following possible redirects, and including unavailable resources in the result set.

\paragraph{Undefined SKOS Resources}

% Origin and rationale
The SKOS model is defined within the namespace \url{http://www.w3.org/2004/02/skos/core#}. However, some vocabularies use resources from within this namespace, which are unresolvable for two main reasons: vocabulary creators ``invented'' new terms within the SKOS namespace instead of introducing them in a separate namespace, or they use ``deprecated'' SKOS properties.
% Function definition
Undefined SKOS resources can be identified by iterating over all resources in $IR$ and returning those resources (i) that are contained in the list of deprecated resources or (ii) are identified by a URI in the SKOS namespace but are not defined in the current version of the SKOS ontology.
